\section{Discussion}

\red{What does this study add}
Epilepsy is a poorly understood disorder, and there is uncertainty about the mechanisms involved in seizure generation and termination. However, with computational models of neural recording becoming more descriptive of physiology, it is becoming possible to gain insights into the underlying mechanism involved in recorded EEG. Computational models of EEG are non-linear, and standard estimation techniques cannot be used to approximate the physiological states from these models. In this paper, the application of the unscented Kalman filter to the Wendling model has been considered to help further the understanding of the mechanisms involved in seizure generation and termination.

\red{Why  neural mass model, and is it relevant to physiology}

Are neural mass models, good models of the brain? To answer this question a definition for the word model is required. In this paper a model is considered to be an approximate descriptor of a system, which in this case is the brain. Since neural mass models are derived from aspects of neural physiology they can be considered to be models of the brain. Whether they are good models of the brain depends on how well they are able to replicate the observations they are intended to mimic. That being said, a good neural mass model is merely one of numerous descriptors of the brain, and without further evaluation of the results obtained from the model with physiological studies a model provides little evidence of casual relationships. However, by using these models insight can be gained into what aspects of physiology should be evaluated, and can provide a method for aspects of physiology that can  not normally be observed, to be crudely estimated. For this study, a good model will be considered to be any model that can describe normal and seizure EEG with distinctly different physiological properties. Further, this model should take into account all relevant physiological studies that have demonstrated causal relationships in the brain that would affect the dynamics produced by the model. 

The relevance of computational models of neural masses to real physiology is still an open question, and we do not propose that the results shown here are indicative of actual physiological mechanisms. However, the derivation of the computational model considered is based on physiological observations made in the hippocampus\citep{freeman1963electrical,wilson1973mathematical,white2000networks}, and the ability of the model to mimic recorded EEG~(Figure~\ref{fig: EEG}) from the hippocampus is promising. If we assume that the the model is a good description of the cortical region we record EEG from, then we can at least say that the estimation results provide some insight into what may be the most likely causes of seizures. If we further assume that variability in the underlying structure of recorded EEG is due to different physiological mechanism at play, we can also conclude that variability in the estimation results between, and within, animals is due to different mechanisms of action, and that seizures are indeed specific to each individual animal.

\red{Why estimation (UKF)}

\red{Variability in animal seizures between and within animals}

\red{UKF tracking of model dynamics}

\red{Variability between seizures, is this just an estimation phenomena from initialised states}

\red{Why is this useful}

\red{Shortcomings}

\red{Why estimation}


\red{Relevance to physiology}

The unscented Kalman filter is capable of tracking changes in the dynamics in recorded EEG. This is clear when considering the estimation results~(Figure~\ref{fig: EstimationResults}) where at seizure initiation and termination there are clear changes in the estimated values and the dynamics of all model parameters. Figure~\ref{fig: SZComp} further demonstrates the estimated model parameters provide a good description of the transition from background to seizure. However, there are clear differences between the recorded and simulated EEG. The first reason for this is that the input to the simulated data varies from simulation to simulation because it is stochastic. Secondly, we assume that the stochastic input to the model has a constant variance, which may not necessarily be the case. In fact by increasing the variance to the simulated model recorded seizure activity is  more accurately replicated; however, pre-seizure and post seizure EEG are less accurately mimicked.

The reliability of the estimates of each model parameter can be seen by considering the pre-seizure results~(Figure~\ref{fig: EstimationResults}). For animal one the pre-seizure estimates for all parameters, except slow inhibition, converge to similar values. This provides good evidence that the estimation is indeed extracting the key characteristics of the recorded EEG. Further, for animal one the estimated model parameters during seizure converge to similar values and have very similar trends. The post seizure results for slow inhibition and excitation are similar between seizures; this is not the case for the input mean and fast inhibition. We observe similar results in all animals, with a few exceptions. For animal two pre-seizure results for all parameters except fast inhibition are similar, there is variability in the values the algorithm converges to during seizures; however, the trends in all estimated parameters for each seizure are alike. Animal 3 has very similar results to animal two in terms on parameters that converge to similar values and the trends in the model parameters. 

The results for animal four are different to the results from the other three animals. There is variability in all the estimated model parameters for all periods shown. The only parameter that converges to similar values between seizures is the excitory synaptic gain pre-seizure. Of particular interest is the estimation results from seizure three, here the model parameters converge to values that are physiologically implausible. There, are two possible reasons for this result: firstly, divergence in the unscented Kalman filter; secondly, model inadequacy when considering this particular seizure. 

The estimated model parameters for each animal are different~(Figure~\ref{fig: SZComp}). This is clear when considering the estimated results for the seizure period. However, the results from pre-seizure and post seizure also show different mechanisms of action. This results may indicate that there are different regulatory mechanisms that occur in the brain in order to suppress seizures, or that there are variations in the networks structure recorded from in the different animals. Another possible reason may be slightly different electrode locations. For the seizure period there are clear differences between the estimation results between animals. For animals one to three the excitatory and fast inhibitory synaptic gain, as well as the input mean is always larger than it is prior to seizure. This is not the case for animal four where the excitatory and fast inhibitory synaptic gain as well as the input mean decreases at seizure initiation. The variation between seizures both in terms of trends and values converged to, is clear for the slow inhibitory synaptic gain. The results from animal one and four are more similar than the results for animals two and three. 

The results from the estimation procedure above suggest that the mechanisms involved in seizure may vary between animals, and possibly within animals over longer time periods. This contradicts the majority of studies that have shown single parameter sets that are capable of describing the transition from background to seizure~\citep{wendling2005interictal}. The reason for this is two fold: the first is that the majority of studies attempt to match the frequency of the observed EEG to the simulated model, whereas the unscented Kalman filter approach makes uses each individual recording and its covariance to determine the most likely parameter values that describe each observation. The second reason is that the unscented Kalman filter makes the assumption that the model is not capable of completely describing the observed EEG (model uncertainty), this uncertainty incorporates model inadequacy as well as the stochastic effect of the input on the model output. In this way the unscented Kalman filter is more robust at dealing with stochastic inputs.

The results comparing the standard deviation and mean of a single simulation compared to a Monte-Carlo simulation demonstrate that the unscented Kalman filter characterizes the expected error on each parameter with some accuracy~(Figure~\ref{fig: MonteResults}). In particular, when considering the slow inhibitory gain there is a high standard deviation in both the single and multiple estimation results prior to seizure. The standard deviation then decreases during seizure and slow increase post seizure. The results also show that the estimation technique is robust to the initialized parameter values. However, one interesting aspect to notice is the high standard deviation of the slow inhibitory gain prior to seizure. This is of particular interest as the estimation results from four seizures from this animal showed variations in the slow inhibitory synaptic gain prior to seizure. The variation between these estimated parameters appear to be due to the particular initializations of the model parameters.

\red{Reason for lower standard deviation for monte carlo analysis during seizure periods}
This may be due to a higher signal to noise ratio in seizure periods. Another possible reason is that this model was developed to mimic seizure dynamics. Therefore, in seizure periods there is less uncertainty about the model predictions then there would be in background EEG.
\red{end}

\red{Reason for high standard deviation post seizure, monte carlo}
 This is due to the nature of the model, and the estimation results for the excitory synaptic gain. Post seizure the excitory synaptic gain is lower than any other period in the estimation procedure. When looking at the structure of the model, it is clear that when the excitory synaptic gain is low the relative contribution of all other populations to the model output is low. With the correction step of the unscented Kalman filter the contribution of these populations would be seen to be similar to noise on the observations, therefore, there is higher uncertainty about the estimates. The reason for a high variance on the input is apparent, as it is the mean of a stochastic element. 
\red{end}

The Kalman filter is a Markov process, so there is an assumption that all the required information to determine the most likely states at the next time point can be compressed into the mean and covariance of the current states. Further, the greater the number of states that are used in the unscented Kalman filter framework the more likely the algorithm is to diverge from the values that best describe the observations. 