\section{Discussion}

\red{Why estimation}
Epilepsy is a poorly understood disorder, and there is uncertainty about the mechanisms involved in seizure generation and termination. However, with computational models of neural recording becoming more descriptive of physiology, it is becoming possible to gain insights into the underlying mechanism involved in recorded EEG. Computational models of EEG are non-linear, and standard estimation techniques cannot be used to approximate the physiological states from these models. In this paper, the application of the unscented Kalman filter to the neural mass model of the hippocampus has been considered.

\red{Relevance to physiology}

The relevance of computational models of neural masses to real physiology is still an open question, and we do not propose that the results shown here are indicative of actual physiological mechanisms. However, the derivation of the computational model considered is based on physiological observations made in the hippocampus, and the ability of the model to mimic recorded EEG~(Figure~\ref{fig: EEG}) from the hippocampus is promising. If we assume that the the model is a good description of the cortical region we record EEG from, then we can at least say that the estimation results provide some insight into what may be possible the most likely causes of seizures. If we further assume that variability in the underlying structure of recorded EEG is due to different physiological mechanism at play, we can also conclude that variability in the estimation results between, and within, animals is due to different mechanisms of action, and that seizures are indeed specific to each individual animal.

\red{Estimation results}

The unscented Kalman filter is capable of tracking changes in the dynamics in recorded EEG. This is clear when considering the estimation results~(Figure~\ref{fig: EstimationResults}) where at seizure initiation and termination there are clear changes in the estimated values or the dynamics of all model parameters. Figure~\ref{fig: SZComp} further demonstrates the estimated model parameters provide a good description of the transition form background to seizure. However, there are clear differences between the recorded and simulated EEG. The first reason for this is that the input to the simulated data varies from simulation to simulation because it is stochastic. Secondly, we assume that the stochastic input to the model has a constant variance, which may not necessarily be the case. In fact by increasing the variance to the simulated model recorded seizure activity is  more accurately replicated; however, preseizure and post seizure EEG are less accurately mimicked.

The reliability of the estimates of each model parameter can be seen by considering the preseizure results~(Figure~\ref{fig: EstimationResults}). For animal one the preseizure estimated for all parameters, except slow inhibition, converge to similar values. This provides good evidence that the estimation is indeed extracting the key characteristics of the recorded EEG. Further, for animal one the estimated model parameters during seizure converge to similar values and have very similar trends. The post seizure results for slow inhibition and excitation are similar between seizures; this is not the case for the input mean and fast inhibition. We observe similar results in all the three other animals, with a few exceptions. For animal two preseizure results for all parameters except fast inhibition are similar, there is variability in the values the algorithm converges to during seizures; however, the trends in all estimated parameters fro each seizure are similar. Animal 3 has very similar results to animal two in terms on parameters that converge to similar values and the trends in the model parameters. 

The results for animal four are different to the results from the other three animals. There is variability in all the estimated model parameters for all periods shown. The only parameter that converges to similar values between seizures is the excitatory synaptic gain preseizure. Of particular interest is the estimation results fro seizure three, which converge to values that are physiologically implausible. There, are two possible reasons for this result: firstly, divergence in the unscented Kalman filter; secondly, model inadequacy when considering this particular seizure. 

The estimated model parameters for each animal are different~(Figure~\ref{fig: SZComp}). This is clear when considering the estimated results for the seizure period. However, the results from preseizure and post seizure also show different mechanisms of action. This results may indicate that there are different regulatory mechanisms that occur in the brain in order to suppress seizures, or that there are variations in the networks structure recorded from in the different animals. Another possible reason may be slightly different electrode locations. For the seizure period there are clear differences between the estimation results between animals. For animals one to three the excitatory synaptic gain is always larger than it is preseizure. This is not the case for animal four where the excitatory synaptic gain decreases at seizure initiation. 















