%%Created by Richard Balson
%%%%%%%%%%%%%%%%%%%%%%%%%%%%%%%%%%%%%%%%%%%%%%%%%%%%%%%%%%%%%

%------------------------------------------------------------
%
\documentclass[10pt]{article}%
%Options -- Point size:  10pt (default), 11pt, 12pt
%        -- Paper size:  letterpaper (default), a4paper, a5paper, b5paper
%                        legalpaper, executivepaper
%        -- Orientation  (portrait is the default)
%                        landscape
%        -- Print size:  oneside (default), twoside
%        -- Quality      final(default), draft
%        -- Title page   notitlepage, titlepage(default)
%        -- Columns      onecolumn(default), twocolumn
%        -- Equation numbering (equation numbers on the right is the default)
%                        leqno
%        -- Displayed equations (centered is the default)
%                        fleqn (equations start at the same distance from the right side)
%        -- Open bibliography style (closed is the default)
%                        openbib
% For instance the command
%           \documentclass[a4paper,12pt,leqno]{article}
% ensures that the paper size is a4, the fonts are typeset at the size 12p
% and the equation numbers are on the left side
%
\usepackage{amsmath}%
\usepackage{amsfonts}%
\usepackage{amssymb}%
\usepackage{graphicx}
\usepackage{hyperref}
\usepackage[round]{natbib}
\usepackage{geometry}
\usepackage{color}
\usepackage{setspace}
\usepackage{pdfsync}
\usepackage{subfig}
\usepackage[export]{adjustbox}
%\usepackage{caption}
%\usepackage{subcaption}
\doublespacing

\def\imagetop#1{\vtop{\null\hbox{#1}}}

%-------------------------------------------
\newenvironment{proof}[1][Proof]{\textbf{#1.} }{\ \rule{0.5em}{0.5em}}

\geometry{top=3cm,bottom=3cm,left=2.5cm,right=3.5cm}

% Text layout
\topmargin 0.0cm
\oddsidemargin 0.5cm
\evensidemargin 0.5cm
\textwidth 16cm 
\textheight 21cm

% Bold the 'Figure #' in the caption and separate it with a period
% Captions will be left justified
\usepackage[labelfont=bf,labelsep=period,justification=raggedright]{caption}

% \newcommand\iref{\textcolor{red}{ref}}
\newcommand\red{\textcolor{red}}

\graphicspath{{fig/}}

\hypersetup{colorlinks,linkcolor=black,filecolor=black,urlcolor=black,citecolor=black} 

\newcommand{\rich}[1]{\textcolor{green}{#1}}
\newcommand{\dean}[1]{\textcolor{blue}{#1}}

\title{Tracking Physiological Changes in an In-Vivo Model of Epilepsy: A Model-Based Approach}
\date{\today}
\author{Richard S. Balson, Dean R. Freestone, Anthony N. Burkitt, Mark J. Cook \\ and David B. Grayden}

\begin{document}
\maketitle

\section{Introduction}

\red{Estimate model parameters from a neural mass model of the hippocampus.}

\red{To improve the understanding of epilepsy} 

\red{How has this been achieved previously?}

\red{Introduction to neural mass model, freeman, jansen etc}

\red{What is the output of the model and why}

\red{Is a neural mass model a good model}

\red{Inadequacy of jansen model and intro to the wendling model}

\red{This model is capable of replicating key characteristics observed in EEG prior to and during seizure.}

\red{Is the Wendling model a good model}

\red{Previous work on estimating the neural mass model of the hippocampus has been done using a genetic algorithm.}

\red{Estimation of the neural mass model (Genetic Algorithm)}

\red{Kalman filter}

\red{Why the Kalman filter}

\red{What is being done, and why is it better or different?}

\red{What is being done with the UKF}

\red{Structure of the paper}

\red{Model estimation and accuracy for real data}

\section{Methods}

\subsection{Animal Model of Epilepsy}

In this study we make use of an \textsl{in vivo} model of temporal lobe epilepsy. In particular, we consider the tetanus toxin model of temporal lobe epilepsy, where the hippocampus is the seizure focus. This particular model was chosen as it has a localised seizure focus, and results in the animals having seizure events that are similar in terms of electrophysiology and clinical aspects to human patients with temporal lobe epilepsy. 

For this particular \textsl{in vivo} model, Sprague Dawley rats that weigh between 300-400 grams are used, with the approval of an ethics committee. Tetanus toxin is injected into the hippocampus at the location of -3.5mm anterior-posterior (AP), 3mm medial lateral and -3.5mm dorsal ventral (DV). This location is within the CA3 region of the hippocampus (Figure~\ref{fig: TetLoc}). Two depth electrodes are placed in the same location to record local field potentials from the seizure focus. A further three burr holes are made, one over the cerebellum, and two over the right hemisphere of the brain. Three electrodes are placed in the burr holes: the cerebellar electrode is used as a reference for all differential recordings, and the two electrodes over the right hemisphere are used to record local filed potentials from the surface of the brain. These electrodes are then connected to a pedestal and held in place by dental cement. The animals are left to recover for seven days, and are then recorded from 24 hours a day for six weeks.

 

\red{ What we are doing in this paper}


\subsection{Neural Mass Model} 

All neural field models can be described by a temporal and spatial response curve, and a function that relates membrane potentials to firing rates. For neural mass models the spatial element is removed from this general description such that

\begin{equation}\label{eq:conv_eq}
    v_n(t) = \frac{\alpha_{mn}}{\tau_{mn}}\int_{-\infty}^t  h_{mn}(t-t')g(v_m(t')) \,\mathrm{d}t' + v_r,
\end{equation}
where $\alpha_{mn}$ is a measure of the connectivity and synaptic strength between populations $m$ and $n$, $h_{mn}(t)$ is the temporal response curve, commonly referred to as a post synaptic response kernel, $\tau_{mn}$ is the time constant which indicates the average prorogation delay between populations $m$ and $n$ and $v_r$ is the resting membrane potential. The temporal response curve is a time delayed exponential decay function:
\begin{equation}
    h_{mn}(t) = \eta(t)t\exp\left(-\frac{t}{\tau_{mn}}\right),
\end{equation}
where $\eta(t)$ is the Heaviside step function. An example of a post synaptic response kernel can be seen in Figure~\ref{fig: Simple}


The firing rate of each population is determined from an asymmetric sigmoid activation curve. The sigmoid function translates the membrane potential of a population to its effective firing rate.
\begin{align}\label{eq:sigmoid}
    g\left(v_n(t)\right) =& \frac{1}{1+\exp{\left(\varsigma_n\left(v_{0n} - v_n(t)\right)\right)}}.
\end{align}
The quantity $\varsigma_n$ describes the gradient of the sigmoidal activation function and $v_{0n}$ the membrane potential that results in half the maximum firing rate. The constant $v_{0n}$ is a lumped parameter, that includes the effect of the resting membrane potential on the activation curve. An example of the sigmoidal activation function is demonstrated in Figure~\ref{fig: Simple}.

 \dean{Write something about $v_r$ being lumped into $v_0$.}

The convolution in Equation~\ref{eq:conv_eq} can also be written as the ordinary differential equation (ODE)
\begin{equation}\label{eq:2ndOrder}
    \mathrm{D}v_n(t) = \frac{\mathrm{d}^2 v_n(t)}{\mathrm{d}t^2} + \frac{2}{\tau_{mn}}\frac{\mathrm{d} v_n(t)}{\mathrm{d}t} + \frac{1}{\tau_{mn}^2} v_n(t) = \frac{\alpha_{mn}}{\tau_{mn}} g(v_m(t)),
\end{equation}
where $\mathrm{D}$ is a differential operator. Equation~\ref{eq:2ndOrder} can be written as two partial differential equations
\begin{equation} \label{eq:2ndOrderNMM}
    \frac{\mathrm{d} v_n(t)}{\mathrm{d}t} = z_n(t),\,\,\,\,\,    \frac{\mathrm{d}z_n(t)}{\mathrm{d}t} = \frac{\alpha_{mn}}{\tau_{mn}} g(v_m(t)) - \frac{2}{\tau_{mn}}z_n(t) - \frac{1}{\tau_{mn}^2} v_n(t),
\end{equation}
where $z_n(t)$ is the time derivative of $v_n(t)$. This forms the basis of a state-space model for all neural masses.

The simplest possible neural mass model is one that takes an input firing rate, and outputs a firing rate. A model of this is shown in Figure~\ref{fig: Simple}, where the effect of both the post synaptic response kernel and the sigmoidal activation curve are demonstrated.

\begin{figure}
	\centering
		\includegraphics{Biological_Response.pdf}
	\caption{Simple Neural Mass Model. (\textbf{A}) An example of a simple neural mass model. This model consists of a single neural population - the triangle - and a single synapse - the square block. (\textbf{B}) Temporal response curve for a synapse. The black line demonstrates the response of a synapse to an impulse response. The time constant ($\tau$) and synaptic gain ($\alpha$) are demonstrated in this figure. The peak response of this function is $\alpha \exp(-1)$ and the time at which this peak occurs is equal to $\tau$. \textbf{C} Somatic response curve. The black line indicates the function $g(v)$ which is a sigmoidal activation curve. This function demonstrates the resulting normalized firing rate produced by a neural population as a function of the membrane potential over it. In this figure $v_{0}$ is the membrane potential at which the firing rate produced is half its maximum.}
	\label{fig: Simple}
\end{figure}

 \begin{figure}
 	\centering
 		\includegraphics{fig/Biological_Model.pdf}
 	\caption{Graphical Description of the Neural Mass Model of the Hippocampus. (\textbf{A}) Graphical description of the neural mass model of the hippocampus. In this figure the traingle, circle diamond and hexagon represent populations of pyramidal, spiny stellate and slow and fast inhbitory neurons. Synaptic connections are specified by a u shaped structure, where the secondary terminal has different shapes depending on the response of the synapse. The 'input' specified in the figure is a firing rate from other cortical regions connected to the modelled area. (\textbf{B}) Flow diagram of the of the neural mass model of the hippocampus. (\textbf{C}) Simplified graphical description of the neural mass model of the hippocampus.}
 	\label{fig: Biological}
 \end{figure}


\subsection{Estimation}

%\red{Generic description on a nonlinear system}
%
%A generic nonlinear system can be defined in terms of a state and observation function \begin{align}
%\label{eqn: NonlinEstS}
%\mathbf{\dot{x}}(t) &= \mathbf{A}(\mathbf{x}(t),\mathbf{\theta}(t)) + \mathbf{B}(\mathbf{u}(t)) + \mathbf{n}(t)\\
%\label{eqn: NonlinEstO}
%\mathbf{y}(t)  &= \mathbf{C}(\mathbf{x}(t)) +\mathbf{D}(\mathbf{u}(t))+\mathbf{r}(t),
%\end{align} where boldface is used to indicate a matrix or vector. Fr the case of a nonlinear system boldface notation can also indicate a matrix of functions. Here, $\mathbf{x}(t)$ is the state vector and $\dot{\mathbf{x}}(t)$ is its derivative, where
%\[ \mathbf{\dot{x}}(t) = \left[ \begin{array}[pos]{c}
%\dot{x}_{1}(t)\\
%\vdots \\
%\dot{x}_{n}(t) \end{array} \right] .\] $\mathbf{A}$ and $\mathbf{B}$ are the state and input matrices, respectively and $\mathbf{u}(t)$ is the input to the model. $\mathbf{C}$ and $\mathbf{D}$ are the output and input-to-output matrices, respectively. The output of the model is $\mathbf{y}(t)$ with $\mathbf{n}(t)$ the model uncertainty, and $\mathbf{r}(t)$ the observation noise. Here, $\mathbf{n}(t)$ is a bound on the expected model error, and $\mathbf{r}(t)$ describes the noise in the observations. Both $\mathbf{n}(t)$ and $\mathbf{r}(t)$ are zero mean Gaussian distributed with a system dependant variance. 


\red{Introduction to the UKF}

Estimation techniques attempt to approximate the model states given some observation. There are numerous techniques to achieve this, most of which are for linear systems. In this paper we make use of a nonlinear estimation technique, the unscented Kalman filter. This method consists of two steps: prediction and correction. The prediction step makes use of an unscented transform (Figure~\ref{fig: UKF}) to approximate the mean and covariance of the model states after they have been propagated through the system. The correction step uses this information to determine the most likely states that best describes the both the observation and the prediction. This step makes use of the expected error on both the observations and the states to determine a weighting for each (Figure~\ref{fig: UKF}).

 \begin{figure}
 	\centering
 		\includegraphics{fig/UnscentedKalman.pdf}
 	\caption{The unscented Kalman filter. (\textbf{A}) The unscented transform. In this figure the a Gaussian distribution of a two dimensional state space is shown. Points one standard deviation  from the mean (red star) are drawn from this distribution (blue stars) and propagated through the system ($f(x)$). The resulting points and the estimated Gaussian distribution of the states is then approximated. (\textbf{B}) The Kalman filter. The prior distribution in the figure is the result from the unscented transform. The Kalman filter predicts the most likely state (Posterior) and its expected error making use of the knowledge of the distributions of the predicted state from the unscented transform and the observation (Obs.). This result is then used in the next iteration of the unscented transform.}
 	\label{fig: UKF}
 \end{figure}

%\red{The unscented filter}

%The unscented filter is completely described by a state's mean and covariance such that:\begin{align}
%\label{eqn: Unscented_Transform1}
%\mathbf{\mathcal{X}}_{n} &= \mathbf{\overline{x}}_{k} + (\sqrt{\kappa+D_{x}\mathbf{P_{xx,k}}})_{n} \quad n=1,\hdots,D_x\\
%\label{eqn: Unscented_Transform2}
%\mathbf{\mathcal{X}}_{n+D_{x}} &= \mathbf{\overline{x}}_{k} - (\sqrt{\kappa+D_{x}\mathbf{P_{xx,k}}})_{n} \quad n=1,\hdots,D_x,\\
%\label{eqn: Unscented_TransformY1}
%\mathbf{\mathcal{Y}}_{n} &= \mathbf{\overline{y}}_{k} + (\sqrt{\kappa+D_{y}\mathbf{P_{yy,k}}})_{n} \quad n=1,\hdots,D_y\\
%\label{eqn: Unscented_TransformY2}
%\mathbf{\mathcal{Y}}_{n+D_{y}} &= \mathbf{\overline{y}}_{k} - (\sqrt{\kappa+D_{y}\mathbf{P_{yy,k}}})_{n} \quad n=1,\hdots,D_y,
%\end{align} where $\mathbf{\overline{x}}_{k}$ and $\mathbf{\overline{y}}_{k}$ are the current state estimate and observation. $(\sqrt{\cdot})_{n}$ denotes the $n$th row of the matrix square root. $D_{x}$ and $D_{y}$ indicate the number of states in the system and the number of observations. Covariance matrices, $\mathbf{P_{xx,k}}$ and $\mathbf{P_{yy,k}}$, are the expected error of the current state and observation. The points $\mathbf{\mathcal{X}}$ are called sigma points and represent the states one standard deviation away for the estimated mean. The term $\kappa$ is a predefined constant, which determines the relative effect of the propagation of the mean. If $\kappa$ is equal to zero then the system mean is not propagated as a sigma point. However, if $\kappa$ is greater than zero then \begin{align}
%\mathbf{\mathcal{X}}_{0} &= \mathbf{\overline{x}}_{k}.
%\end{align} Therefore, 2$D_{x}$ sigma points are assigned when $\kappa$ is zero and 2$D_{x}$+1 sigma points are assigned when it is greater than zero. The sigma points are propagated through the system in order to update the expectation about the state mean and error: \begin{align}%%%%%%%%%%%%%%%%%%%%%%%%%%%%%%%%%%%%
%\mathbf{\mathcal{X}}_{n,k+1} &= \mathbf{\mathcal{X}}_{n,k}+ T(\mathbf{A}(\mathbf{\mathcal{X}}_{n,k}) +\mathbf{B}(\mathbf{u}_{k})) +\sqrt{T}{n}_{k}\\
%\overline{\mathbf{x}}_{k+1}^{-} &= \frac{1}{2D_{x}+\kappa}\sum_{n=1}^{2D_{x}} \mathbf{\mathcal{X}}_{n,k+1}\\
%\mathbf{P}_{xx,k+1}^{-} &= \frac{1}{2D_{x}+\kappa}\sum_{n=1}^{2D_{x}} (\mathbf{\mathcal{X}}_{n,k+1} -\mathbf{\overline{x}}_{k+1}^{-})(\mathbf{\mathcal{X}}_{n,k+1}-\mathbf{\overline{x}}_{k+1}^{-})^{\top} + \mathbf{Q}.%%%%%%%%%%%%%%%%%%%%%%%%%%%%%%%%%%%%
%\end{align} $\overline{\mathbf{x}}_{k+1}^{-}$ and $\mathbf{P}_{xx,k+1}^{-}$ are the predictions for the state and state covariance matrices. The negative superscript is used to indicate an uncorrected prediction. The term $\mathbf{Q}$ is the expectation of the model error $n_{k}$ and $(\cdot)^{\top}$ indicates the transpose. It is now possible to make a prediction about the observation at sample $k+1$ by propagating the sigma points through Equation~(\ref{eqn: YProp}) \begin{align} %%%%%%%%%%%%%%%%%%%%%%%%%%%%
%\mathbf{\mathcal{Y}}_{n,k+1} &= \mathbf{\mathcal{Y}}_{n,k} + T(\mathbf{C}(\mathbf{\mathcal{X}}_{n,k+1})+ \mathbf{D}(\mathbf{u}_{k}))+ \sqrt{T}\mathbf{r}_{k}\\
%\overline{\mathbf{y}}_{k+1}^{-} &= \frac{1}{2D_{x}+\kappa}\sum_{n=1}^{2D_{x}} \mathbf{\mathcal{Y}}_{n,k+1}\\
%\label{eqn: statecovg}
%\mathbf{P}_{xy,k+1}^{-} &= \frac{1}{2D_{x}+\kappa}\sum_{n=1}^{2D_{x}} (\mathbf{\mathcal{X}}_{n,k+1}-\overline{\mathbf{x}}_{n,k+1}) (\mathbf{\mathcal{Y}}_{n,k+1}-\overline{\mathbf{y}}_{k+1}^{-})^{\top}\\
%\mathbf{P}_{yy,k+1}^{-} &= \frac{1}{2D_{x}+\kappa}\sum_{n=1}^{2D_{x}} (\mathbf{\mathcal{Y}}_{n,k+1}-\overline{\mathbf{y}}_{k+1}^{-}) (\mathbf{\mathcal{Y}}_{n,k+1}-\overline{\mathbf{y}}_{k+1}^{-})^{\top} +\mathbf{R},%%%%%%%%%%%%%%%%%%%%%%%%%%%%%%%%%%%%%%%%%%%%
%\end{align} where $\overline{\mathbf{y}}_{k+1}^{-}$ and $\mathbf{P}_{yy,k+1}^{-}$ are the predictions for the model output and its covariance, respectively. $\mathbf{P}_{xy,k+1}^{-}$ is the covariance matrix of the states and observations and $\mathbf{R}$ is the expectation of the observation error $\mathbf{r}_{k}$.


%\red{How states are predicted using the unscented transform}

%The predictions of the states,$\overline{\mathbf{x}}_{k+1}^{-}$, and observations, $\overline{\mathbf{y}}_{k+1}^{-}$, now need to be corrected based on the observations. This is achieved by determining the Kalman gain and updating the predictions based on the current observation: \begin{align}
%\mathbf{K} &= \mathbf{P}_{xy,k+1}^{-}(\mathbf{P}_{yy,k+1}^{-})^{-1}\\
%\overline{\mathbf{x}}_{k+1} &= \overline{\mathbf{x}}_{k+1}^{-} + \mathbf{K}(\mathbf{y}_{k+1}-\overline{\mathbf{y}}_{k+1}^{-})\\
%\mathbf{P}_{xx,k+1} &= \mathbf{P}_{xx,k+1}^{-} - \mathbf{K}(\mathbf{P}_{xy,k+1}^{-})^{\top},
%\end{align} where $\mathbf{y}_{k}$ is the observation, $\overline{\mathbf{x}}_{k+1}$ is the corrected estimate of the state and $\mathbf{P}_{xx,k+1}$ is the estimate of its error. This set of equations describes the UKF and how it can be used to estimate states. However, for this study, estimation of states and parameters is required (dual estimation).


%\red{Definition of slow state matrix and its dynamics}

In this paper we are interested in estimating aspects of physiology that are unobserved by making use of a neural mass model. In particular, we consider the estimation of the model's synaptic gains ($\alpha_{mn}$) from equation~\ref{eq:2ndOrderNMM}. To estimate model parameters they need to be augmented to the original state matrix and given dynamics. This model parameters can be assigned trivial dynamics due to the prediction correction nature of the unscented Kalman filter. By this we mean that the model parameter is assumed to be constant with some error assigned to it. This error allows this model parameter to have slow dynamics, and for it to track its true value. The change in model parameters occurs at a longer time scale than the model states. Therefore, for convenience, the model states and parameters are referred to as fast and slow states. 

\red{Intialisation of UKF for stationary parameters}

\red{Intialisation of UKF for varying parameters}

\red{Estimation of model input mean}

\red{Robustness test}

\section{Results}

\begin{figure}
 	\centering
 		\includegraphics{fig/EEG.pdf}
 	\caption{Recorded and Simulated EEG. (\textbf{A}) Recorded EEG from an \textsl{in vivo} model of temporal lobe epilepsy. (\textbf{B}) Artificial EEG generated from the neural mass model of the hippocampus.}
 	\label{fig: EEG}
 \end{figure}


\begin{figure}
 	\centering
 		\includegraphics{fig/SeizureComparison.pdf}
 	\caption{Estimation Results for Multiple Animals. In the top graph artificial EEG using the neural mass model of the hippocampus is demonstrated. The parameters used for this simulation are the same as the results for seizure(SZ) one in this figure. The results from seizures for four different animals is demonstrated in this figure, each corresponding to a different color line. All seizure initiate at time 0s and end at the dotted line that is the same color as the estimation results for the animal.}
 	\label{fig: EEG}
 \end{figure}

\begin{figure}
 	\centering
	 \vspace*{-2cm}
 		\includegraphics[max size={\textwidth}{\textheight}]{fig/SameInitEstResult.pdf}
 	\caption{Results from the estimation of four seizures from four different animals . (\textbf{A}), (\textbf{B}), (\textbf{C}), (\textbf{D}) Estimation results for four seizures(SZ) from animal 1-4, respectively. The colors in all graphs correspond to the legend above the figure. The first estimated seizure is shown in the top graph, where the initiation of seizure begins at time 0s and ends at the dotted black line. In the corresponding graphs the estimation results for the excitatory ($\alpha_{\mathrm{p}}$), slow inhibitory ($\alpha_{\mathrm{s}}$) and fast inhibitory ($\alpha_{\mathrm{f}}$) synaptic gains are demonstrated, where all seizures initiate at 0s and terminate at the dotted line with the same color. Lastly, the estimate for the mean input firing rate to the modeled population is demonstrated.}
 	\label{fig: EstimationResults}  
 \end{figure}

\section{Discussion}

\red{Why estimation}

\red{Effect of stochastic input}

\red{Estimation of model parameters}

\red{Initialisation error}

\red{Observation Noise}

\red{Parameters varying}

% 
% \section{Discussion}

\red{What does this study add}
Epilepsy is a poorly understood disorder, and there is uncertainty about the mechanisms involved in seizure generation and termination. However, with computational models of neural recording becoming more descriptive of physiology, it is becoming possible to gain insights into the underlying mechanism involved in recorded EEG. Computational models of EEG are non-linear, and standard estimation techniques cannot be used to approximate the physiological states from these models. In this paper, the application of the unscented Kalman filter to the Wendling model has been considered to help further the understanding of the mechanisms involved in seizure generation and termination.

\red{Why  neural mass model, and is it relevant to physiology}

Are neural mass models, good models of the brain? To answer this question a definition for the word model is required. In this paper a model is considered to be an approximate descriptor of a system, which in this case is the brain. Since neural mass models are derived from aspects of neural physiology they can be considered to be models of the brain. Whether they are good models of the brain depends on how well they are able to replicate the observations they are intended to mimic. That being said, a good neural mass model is merely one of numerous descriptors of the brain, and without further evaluation of the results obtained from the model with physiological studies a model provides little evidence of casual relationships. However, by using these models insight can be gained into what aspects of physiology should be evaluated, and can provide a method for aspects of physiology that can  not normally be observed, to be crudely estimated. For this study, a good model will be considered to be any model that can describe normal and seizure EEG with distinctly different physiological properties. Further, this model should take into account all relevant physiological studies that have demonstrated causal relationships in the brain that would affect the dynamics produced by the model. 

The relevance of computational models of neural masses to real physiology is still an open question, and we do not propose that the results shown here are indicative of actual physiological mechanisms. However, the derivation of the computational model considered is based on physiological observations made in the hippocampus\citep{freeman1963electrical,wilson1973mathematical,white2000networks}, and the ability of the model to mimic recorded EEG~(Figure~\ref{fig: EEG}) from the hippocampus is promising. If we assume that the the model is a good description of the cortical region we record EEG from, then we can at least say that the estimation results provide some insight into what may be the most likely causes of seizures. If we further assume that variability in the underlying structure of recorded EEG is due to different physiological mechanism at play, we can also conclude that variability in the estimation results between, and within, animals is due to different mechanisms of action, and that seizures are indeed specific to each individual animal.

\red{Why estimation (UKF)}

\red{Variability in animal seizures between and within animals}

\red{UKF tracking of model dynamics}

\red{Variability between seizures, is this just an estimation phenomena from initialised states}

\red{Why is this useful}

\red{Shortcomings}

\red{Why estimation}


\red{Relevance to physiology}

The unscented Kalman filter is capable of tracking changes in the dynamics in recorded EEG. This is clear when considering the estimation results~(Figure~\ref{fig: EstimationResults}) where at seizure initiation and termination there are clear changes in the estimated values and the dynamics of all model parameters. Figure~\ref{fig: SZComp} further demonstrates the estimated model parameters provide a good description of the transition from background to seizure. However, there are clear differences between the recorded and simulated EEG. The first reason for this is that the input to the simulated data varies from simulation to simulation because it is stochastic. Secondly, we assume that the stochastic input to the model has a constant variance, which may not necessarily be the case. In fact by increasing the variance to the simulated model recorded seizure activity is  more accurately replicated; however, pre-seizure and post seizure EEG are less accurately mimicked.

The reliability of the estimates of each model parameter can be seen by considering the pre-seizure results~(Figure~\ref{fig: EstimationResults}). For animal one the pre-seizure estimates for all parameters, except slow inhibition, converge to similar values. This provides good evidence that the estimation is indeed extracting the key characteristics of the recorded EEG. Further, for animal one the estimated model parameters during seizure converge to similar values and have very similar trends. The post seizure results for slow inhibition and excitation are similar between seizures; this is not the case for the input mean and fast inhibition. We observe similar results in all animals, with a few exceptions. For animal two pre-seizure results for all parameters except fast inhibition are similar, there is variability in the values the algorithm converges to during seizures; however, the trends in all estimated parameters for each seizure are alike. Animal 3 has very similar results to animal two in terms on parameters that converge to similar values and the trends in the model parameters. 

The results for animal four are different to the results from the other three animals. There is variability in all the estimated model parameters for all periods shown. The only parameter that converges to similar values between seizures is the excitory synaptic gain pre-seizure. Of particular interest is the estimation results from seizure three, here the model parameters converge to values that are physiologically implausible. There, are two possible reasons for this result: firstly, divergence in the unscented Kalman filter; secondly, model inadequacy when considering this particular seizure. 

The estimated model parameters for each animal are different~(Figure~\ref{fig: SZComp}). This is clear when considering the estimated results for the seizure period. However, the results from pre-seizure and post seizure also show different mechanisms of action. This results may indicate that there are different regulatory mechanisms that occur in the brain in order to suppress seizures, or that there are variations in the networks structure recorded from in the different animals. Another possible reason may be slightly different electrode locations. For the seizure period there are clear differences between the estimation results between animals. For animals one to three the excitatory and fast inhibitory synaptic gain, as well as the input mean is always larger than it is prior to seizure. This is not the case for animal four where the excitatory and fast inhibitory synaptic gain as well as the input mean decreases at seizure initiation. The variation between seizures both in terms of trends and values converged to, is clear for the slow inhibitory synaptic gain. The results from animal one and four are more similar than the results for animals two and three. 

The results from the estimation procedure above suggest that the mechanisms involved in seizure may vary between animals, and possibly within animals over longer time periods. This contradicts the majority of studies that have shown single parameter sets that are capable of describing the transition from background to seizure~\citep{wendling2005interictal}. The reason for this is two fold: the first is that the majority of studies attempt to match the frequency of the observed EEG to the simulated model, whereas the unscented Kalman filter approach makes uses each individual recording and its covariance to determine the most likely parameter values that describe each observation. The second reason is that the unscented Kalman filter makes the assumption that the model is not capable of completely describing the observed EEG (model uncertainty), this uncertainty incorporates model inadequacy as well as the stochastic effect of the input on the model output. In this way the unscented Kalman filter is more robust at dealing with stochastic inputs.

The results comparing the standard deviation and mean of a single simulation compared to a Monte-Carlo simulation demonstrate that the unscented Kalman filter characterizes the expected error on each parameter with some accuracy~(Figure~\ref{fig: MonteResults}). In particular, when considering the slow inhibitory gain there is a high standard deviation in both the single and multiple estimation results prior to seizure. The standard deviation then decreases during seizure and slow increase post seizure. The results also show that the estimation technique is robust to the initialized parameter values. However, one interesting aspect to notice is the high standard deviation of the slow inhibitory gain prior to seizure. This is of particular interest as the estimation results from four seizures from this animal showed variations in the slow inhibitory synaptic gain prior to seizure. The variation between these estimated parameters appear to be due to the particular initializations of the model parameters.

\red{Reason for lower standard deviation for monte carlo analysis during seizure periods}
This may be due to a higher signal to noise ratio in seizure periods. Another possible reason is that this model was developed to mimic seizure dynamics. Therefore, in seizure periods there is less uncertainty about the model predictions then there would be in background EEG.
\red{end}

\red{Reason for high standard deviation post seizure, monte carlo}
 This is due to the nature of the model, and the estimation results for the excitory synaptic gain. Post seizure the excitory synaptic gain is lower than any other period in the estimation procedure. When looking at the structure of the model, it is clear that when the excitory synaptic gain is low the relative contribution of all other populations to the model output is low. With the correction step of the unscented Kalman filter the contribution of these populations would be seen to be similar to noise on the observations, therefore, there is higher uncertainty about the estimates. The reason for a high variance on the input is apparent, as it is the mean of a stochastic element. 
\red{end}

The Kalman filter is a Markov process, so there is an assumption that all the required information to determine the most likely states at the next time point can be compressed into the mean and covariance of the current states. Further, the greater the number of states that are used in the unscented Kalman filter framework the more likely the algorithm is to diverge from the values that best describe the observations. 
% 
% \section{Conclusion}

In this paper we have demonstrated that the tracking of neural dynamics using an unscented Kalman filter is feasible. We have also shown that there are clear differences between the mechanisms involved in seizure initiation and termination between and in some cases within animals. This information could be used to help with the development of new therapies. Some of the aspects that were not discussed in this paper include variability due to severity of seizures, and the long term evolution of seizures in an animal model where seizure frequency decreases over time. 
% 
% \section{Appendix}
\label{sec: AppendixA}

The reduction of the Wendling model is based on the principles of linear superposition. First notice that each synapse is a linear lowpass filter (equation~\ref{eqn: LaplaceNMM}). In figure~\ref{fig: Biological} there are eight such synapse. However, notice that numerous of these synapse have identical synaptic gains, and are connected by different connectivity constants. Given that for a linear system it is known that
\begin{align}
f(a\mathbf{x}) = af(\mathbf{x}),
\end{align} where $f(\cdot)$ is an arbitrary function, and $a$ is a constant. Therefore, the model can be simplified by moving the synapse closer to the respective populations. The result of this process is demonstrated in figure~\ref{fig: BiologicalMin}.

\begin{figure}  %%%%%%%%%%%%%%%%%%%%%%%%%%%%%%%%%%%%%%%
	\centering
		\includegraphics{Biological_Model_DescriptionMin.png}
	\caption{Graphical Description of the Wendling model (Minimised). Membrane potentials are shown and named $v_{b}$ where $b$ is $p$, $e$, $fi$ and $si$ for pyramidal, excitatory and slow and fast inhibitory populations, respectively. The synaptic gains of each population are specified by $\theta_{m}$ where $m$ is defined in the same manner as the membrane potentials and $\theta_{p}=\theta_{e}$. Each triangle and circle indicates a neural population. In particular, the triangle shape indicates the pyramidal population and the circle shapes represent interneurons. The triangles and circles model the action of the soma for the neural populations, and convert membrane potentials to firing rates. The synapses convert the firing rates to membrane potentials. Lastly, each line indicates a neural connection, which is specified by a connectivity constant.}
	\label{fig: BiologicalMin}
\end{figure}%%%%%%%%%%%%%%%%%%%%%%%%%%%%%%%%%%%%%%%%%%%%%

Now consider the full set of original model equations that would be required to specify figure~\ref{fig: Biological}
\begin{align}
\label{eqn: Fullmodel Descrip}
\mathrm{d}v_{po*}(t)&= z_{po*}(t)\mathrm{d}t\\
\mathrm{d}z_{po*}(t)&=\left(\frac{\theta_{p}(t)}{\tau_{p}}c_{1}g(v_{p}(t))-2\frac{z_{po*}(t)}{\tau_{p}}-\frac{v_{po*}(t)}{\tau_{p}^{2}}\right)\mathrm{d}t\\
\mathrm{d}v_{p1*}(t)&= z_{p1*}(t)\mathrm{d}t\\
\mathrm{d}z_{p1*}(t)&=\left(\frac{\theta_{p}(t)}{\tau_{p}}c_{3}g(v_{p}(t))-2\frac{z_{p1*}(t)}{\tau_{p}}-\frac{v_{p1*}(t)}{\tau_{p}^{2}}\right)\mathrm{d}t\\
\mathrm{d}v_{p2*}(t)&= z_{p2*}(t)\mathrm{d}t\\
\label{eqn: Pyr}
\mathrm{d}z_{p2*}(t)&=\left(\frac{\theta_{p}(t)}{\tau_{p}}c_{5}g(v_{p}(t))-2\frac{z_{p2*}(t)}{\tau_{p}}-\frac{v_{p2*}(t)}{\tau_{p}^{2}}\right)\mathrm{d}t\\
\label{eqn: SI}
\mathrm{d}v_{p3*}(t)&= z_{p3*}(t)\mathrm{d}t\\
\mathrm{d}z_{p3*}(t)&=\left(\frac{\theta_{si}(t)}{\tau_{si}}c_{4}g(v_{si}(t))-2\frac{z_{p3*}(t)}{\tau_{si}}-\frac{v_{p3*}(t)}{\tau_{si}^{2}}\right)\mathrm{d}t\\
\mathrm{d}v_{p4*}(t)&= z_{p4*}(t)\mathrm{d}t\\
\label{eqn: SI1}
\mathrm{d}z_{p4*}(t)&=\left(\frac{\theta_{si}(t)}{\tau_{si}}c_{6}g(v_{si}(t))-2\frac{z_{p4*}(t)}{\tau_{si}}-\frac{v_{p4*}(t)}{\tau_{si}^{2}}\right)\mathrm{d}t\\
\label{eqn: FI}
\mathrm{d}v_{p5*}(t)&= z_{p5*}(t)\mathrm{d}t\\
\label{eqn: FI2}
\mathrm{d}z_{p5*}(t)&=\left(\frac{\theta_{fi}(t)}{\tau_{fi}}c_{7}g(v_{fi}(t))-2\frac{z_{p4*}(t)}{\tau_{fi}}-\frac{v_{p5*}(t)}{\tau_{fi}^{2}}\right)\mathrm{d}t\\
\label{eqn: Exc}
\mathrm{d}v_{p6*}(t)&= z_{p6*}(t)\mathrm{d}t\\
\mathrm{d}z_{p6*}(t)&=\left(\frac{\theta_{e}(t)}{\tau_{e}}c_{2}g(v_{e}(t))-2\frac{z_{p6*}(t)}{\tau_{e}}-\frac{v_{p6*}(t)}{\tau_{e}^{2}}\right)\mathrm{d}t\\
\mathrm{d}v_{p7*}(t)&= z_{p7*}(t)\mathrm{d}t\\
\label{eqn: WienerFull}
\mathrm{d}z_{p7*}(t)&=\left(\frac{\theta_{p}(t)}{\tau_{p}}\mu -2\frac{z_{p7*}(t)}{\tau_{e}}-\frac{v_{p7*}(t)}{\tau_{p}^{2}}\right)\mathrm{d}t + \frac{\theta_{p}(t)}{\tau_{p}}\epsilon(t)\mathrm{d}W.\\
\end{align} In these equations $dW$ represents a Wiener process and is required as $\epsilon(t)\sim N(0,\sigma)$, where $\sigma$ and $\mu$ (equation~\ref{eqn: Wiener}) describe the mean and variance of the stochastic model input. Further, $v_{po}(t) $ and $v_{p1-7*}(t)$ represent the membrane potential produced by each synapse and $z_{po*}(t) $ and $z_{p1-7*}(t)$ their derivatives. The inputs to each neural population are specified by $v_{b}(t) $ and $z_{b}(t) $, and are the membrane potential of the specific population, where $b$ takes the values of $p$, $e$, $fi$ and $si$ representing pyramidal, excitatory, and slow and fast inhibitory populations, respectively. Therefore $v_{p}(t) $ is the output of the model. All $v_{b}(t) $ can be described in terms of $v_{po}(t)$ and $v_{p1-7}(t)$ as follows
\begin{align}
\label{eqn: pop1}
v_{p}(t) &= v_{p6*}(t)+v_{p7*}(t)-v_{p3*}(t)-v_{p5*}(t)\\
\label{eqn: pop2}
v_{e}(t) &= v_{po*}(t)\\
\label{eqn: pop3}
v_{si}(t) &= v_{p1*}(t)\\
\label{eqn: pop4}
v_{fi}(t) &= v_{p2*}(t)-v_{p4*}(t).
\end{align} Notice that equations~\ref{eqn: Fullmodel Descrip}-\ref{eqn: Pyr} and equations~\ref{eqn: SI}-\ref{eqn: SI1} are identical except for the connectivity constants. Further, in this model it is assumed that 
\begin{align}
\theta_p = \theta_e\\
\tau_p = \tau_e.
\end{align} Therefore, equations~\ref{eqn: Exc}-\ref{eqn: WienerFull} are identical, but have different inputs. Next consider the effect of the states $v_{p0-7*}(t)$ and $z_{p0-7*}(t)$ on the model inputs. Note that since the effect of $v_{p6*}(t)$ and $v_{p7*}(t)$ are additive, they can be added prior to being passed through the linear filter since
\begin{align}
f{x} + f{y} = f{x+y},
\end{align} where f is defined above and $x$ and $y$ are arbitrary variables. Therefore equations~\ref{eqn: Exc}-\ref{eqn: WienerFull} can be simplified to the following:
\begin{align}
\mathrm{d}v_{p1}(t)&= z_{p1}(t)\mathrm{d}t\\
\label{eqn: Wiener1}
\mathrm{d}z_{p1}(t)&=\left(\frac{\theta_{e}(t)}{\tau_{e}}(\mu +n_{e}g(v_{e}(t))-2\frac{z_{p1}(t)}{\tau_{e}}-\frac{v_{p1}(t)}{\tau_{e}^{2}}\right)\mathrm{d}t + \frac{\theta_{e}(t)}{\tau_{e}}\epsilon(t)\mathrm{d}W.
\end{align}. By making this reduction the effect of both the excitatory population and the input have been incorporated in one potential. Therefore, $v_{p7*}(t)$ is no longer required in equations~\ref{eqn: pop1}. 

Next consider the equations that only have connectivity constants that are different. Notice that these connectivity constants scale the input to the function, and this scaling can be performed after the input is transformed by the linear function. Therefore, equations~\ref{eqn: Fullmodel Descrip}-\ref{eqn: Pyr} can be represented by a single equation such that
\begin{align}
\mathrm{d}v_{po}(t)&= z_{po}(t)\mathrm{d}t\\
\mathrm{d}z_{po}(t)&=\left(\frac{\theta_{p}(t)}{\tau_{p}}g(v_{p}(t))-2\frac{z_{po}(t)}{\tau_{p}}-\frac{v_{po}(t)}{\tau_{p}^{2}}\right)\mathrm{d}t.
\end{align} Notice that in doing this equations~\ref{eqn: pop2}-\ref{eqn: pop4} are no longer valid and need to be altered to include the connectivities that have been removed from their relevant functions
\begin{align}
\label{eqn: pop2n}
v_{e}(t) &= c_{1}v_{po}(t)\\
\label{eqn: pop3n}
v_{si}(t) &= c_{3}v_{p0}(t)\\
\label{eqn: pop4n}
v_{fi}(t) &= c_{5}v_{p0}(t)-v_{p4*}(t).
\end{align}. 

A similar argument can be applied to equations~\ref{eqn: SI}-\ref{eqn: SI1} which can be simplified to
\begin{align}
\mathrm{d}v_{p2}(t)&= z_{p2}(t)\mathrm{d}t\\
\mathrm{d}z_{p2}(t)&=\left(\frac{\theta_{si}(t)}{\tau_{si}}g(v_{si}(t))-2\frac{z_{p2}(t)}{\tau_{si}}-\frac{v_{p2}(t)}{\tau_{si}^{2}}\right)\mathrm{d}t.
\end{align} Again by doing so equation~\ref{eqn: pop1} and~\ref{eqn: pop4n} are no longer valid and become
\begin{align}
v_{p}(t) &= v_{p1}(t)-c_{4}v_{p2}(t)-v_{p5*}(t)\\
v_{fi}(t) &= c_{5}v_{p0}(t)-c_{6}v_{p2}(t).
\end{align}. Lastly for simplicity equations~\ref{eqn: FI}-\ref{eqn: FI2} are replaced with
\begin{align}
\mathrm{d}v_{p3}(t)&= z_{p3}(t)\mathrm{d}t\\
\label{eqn: FI1}
\mathrm{d}z_{p3}(t)&=\left(\frac{\theta_{fi}(t)}{\tau_{fi}}c_{7}g(v_{fi}(t))-2\frac{z_{p3}(t)}{\tau_{fi}}-\frac{v_{p3}(t)}{\tau_{fi}^{2}}\right)\mathrm{d}t.
\end{align} Which results in the set of equations demonstrated in the methods section. Notice that the $n_{b}$ term used in the general form of the equations specifies connectivities that were not removed from the full model description equations due to simplification.



\bibliographystyle{apalike}
\bibliography{RealData}


\end{document}

