%%Created by Richard Balson
%%%%%%%%%%%%%%%%%%%%%%%%%%%%%%%%%%%%%%%%%%%%%%%%%%%%%%%%%%%%%

%------------------------------------------------------------
%
\documentclass[10pt]{article}%
%Options -- Point size:  10pt (default), 11pt, 12pt
%        -- Paper size:  letterpaper (default), a4paper, a5paper, b5paper
%                        legalpaper, executivepaper
%        -- Orientation  (portrait is the default)
%                        landscape
%        -- Print size:  oneside (default), twoside
%        -- Quality      final(default), draft
%        -- Title page   notitlepage, titlepage(default)
%        -- Columns      onecolumn(default), twocolumn
%        -- Equation numbering (equation numbers on the right is the default)
%                        leqno
%        -- Displayed equations (centered is the default)
%                        fleqn (equations start at the same distance from the right side)
%        -- Open bibliography style (closed is the default)
%                        openbib
% For instance the command
%           \documentclass[a4paper,12pt,leqno]{article}
% ensures that the paper size is a4, the fonts are typeset at the size 12p
% and the equation numbers are on the left side
%
\usepackage{amsmath}%
\usepackage{amsfonts}%
\usepackage{amssymb}%
\usepackage{graphicx}
\usepackage{hyperref}
\usepackage[round]{natbib}
\usepackage{geometry}
\usepackage{color}
\usepackage{setspace}
\usepackage{pdfsync}
\usepackage{subfig}
\usepackage[export]{adjustbox}
%\usepackage{caption}
%\usepackage{subcaption}
\doublespacing

\def\imagetop#1{\vtop{\null\hbox{#1}}}

%-------------------------------------------
\newenvironment{proof}[1][Proof]{\textbf{#1.} }{\ \rule{0.5em}{0.5em}}

\geometry{top=3cm,bottom=3cm,left=2.5cm,right=3.5cm}

% Text layout
\topmargin 0.0cm
\oddsidemargin 0.5cm
\evensidemargin 0.5cm
\textwidth 16cm 
\textheight 21cm

% Bold the 'Figure #' in the caption and separate it with a period
% Captions will be left justified
\usepackage[labelfont=bf,labelsep=period,justification=raggedright]{caption}

% \newcommand\iref{\textcolor{red}{ref}}
\newcommand\red{\textcolor{red}}

\graphicspath{{fig/}}

\hypersetup{colorlinks,linkcolor=black,filecolor=black,urlcolor=black,citecolor=black} 

\newcommand{\rich}[1]{\textcolor{green}{#1}}
\newcommand{\dean}[1]{\textcolor{blue}{#1}}

\title{Tracking Physiological Changes in an In-Vivo Model of Epilepsy: A Model-Based Approach}
\date{\today}
\author{Richard S. Balson, Dean R. Freestone, Anthony N. Burkitt, Mark J. Cook \\ and David B. Grayden}

\begin{document}
\maketitle

\section{Introduction}

\red{Estimate model parameters from a neural mass model of the hippocampus.}

\red{To improve the understanding of epilepsy} 

\red{How has this been achieved previously?}

\red{Introduction to neural mass model, freeman, jansen etc}

\red{What is the output of the model and why}

\red{Is a neural mass model a good model}

\red{Inadequacy of jansen model and intro to the wendling model}

\red{This model is capable of replicating key characteristics observed in EEG prior to and during seizure.}

\red{Is the Wendling model a good model}

\red{Previous work on estimating the neural mass model of the hippocampus has been done using a genetic algorithm.}

\red{Estimation of the neural mass model (Genetic Algorithm)}

\red{Kalman filter}

\red{Why the Kalman filter}

\red{What is being done, and why is it better or different?}

\red{What is being done with the UKF}

\red{Structure of the paper}

\red{Model estimation and accuracy for real data}

\section{Methods}

\red{ What we are doing in this paper}
The Wendling model is capable of mimicking normal and seizure activity observed in hippocampal EEG in people with epilepsy~\citep{wendling2002epileptic,wendling2005interictal}. Here, the estimation of physiologically relevant model parameters from the Wendling model is considered. An unscented Kalman filter (UKF) is used to estimate the model parameters of interest. For this study, the estimation procedure is tested using EEG simulated from the Wendling model to determine the robustness of the UKF.

\subsection{Model Description and Simulation} %%%%%%%%%%%%%%%%%%%%%%%%%%%%%%%%%%%%%%%%%%%%%%%%%%%%%%%%%%%%%%%%%%%%%%%%%%%%%%%%%%%%%%%%%%%%%%%%%%%%%%%%%%%%%%%%%%%%%%%%%%%%%%%%%%%%%%%%%%%%%%%%%%%%%%%%%%%%%%%%%%%%%%%%%%%%%%%%%%%%%%%%%%%%%%%%%%%%%%%%

\red{Wendling model description.}%%%%%%%%%%%%%%%%%%%%%%%%%%%%%
The Wendling model describes the aggregate membrane potentials and firing rates produced by different neural populations. Each neural population is either excited or inhibited by other populations in the model. The net effect of one population on another is determined by a scaling constant termed connectivity. A graphical representation of the model is shown in Figure~\ref{fig: Biological}. In the model, four different neural populations are considered. The pyramidal neural population is the generator of EEG. Excitatory interneurons excite the pyramidal neurons (this connection is often modelled as a time delayed recurrent connection of the pyramidal population). Slow and fast inhibitory interneurons suppress the pyramidal neural population. The pyramidal neural population excites the excitatory and slow and fast inhibitory populations. Slow inhibitory interneurons suppress the fast inhibitory neural population. The effect of each neural population on the other is scaled by connectivity that accounts for the number of afferent synaptic connections between neural populations. A stochastic input to the model is added to account for the unknown effect of afferent pyramidal neurons from other areas of the brain.%%%%%%%%%%%%%%%%%%%%%%%%%
\begin{figure}  %%%%%%%%%%%%%%%%%%%%%%%%%%%%%%%%%%%%%%%
	\centering
		\includegraphics[width = 0.8\textwidth]{Biological1.pdf}
	\caption{Graphical Description of the Wendling model. Membrane potentials are shown and named $v_{b}$ where $b$ is p, e, f and s for pyramidal, excitatory, and fast and slow inhibitory populations, respectively. The synaptic gains of each population are specified by $G_{m}$ where $m$ is defined in the same manner as the membrane potentials and $G_{p}=G_{e}$.The triangle shape indicates the pyramidal population and the circle shapes represent interneurons. Each line indicates a neural connection, which is specified by a connectivity constant.}
	\label{fig: Biological}
\end{figure}%%%%%%%%%%%%%%%%%%%%%%%%%%%%%%%%%%%%%%%%%%%%%

\red{Mathematical functions used in the model} %%%%%%%%%%%%%%%%%%%%%%%%%%%%%%%%%%%%%%%%%%%%%%%%%%%%%%%%%%%%%%%%%%%%%%%%%%%%%%%%%%%%%%%%%%%%%%%%%%%%%%%%%%%%%%%%%%%%%%%%%%%%%%%%%%%%%%%%%%%%%%%
The Wendling model consists of two functions. The first function is a sigmoid function, which converts an aggregate membrane potential to an average firing rate,\begin{align}%%%%%%%%%%%%%%%%%%%%%%%%%%%%%%%%%%%%%%%%%%%%%
\label{eqn: Sigmoid}
g(v(t)) &= \frac{g_{max}}{1+\exp(r(v(t)-v_{0}))}, \end{align} where $g_{max}$ is the maximum the firing rate, $r$ is the sigmoid gradient, and $v_{0}$ is the membrane potential at which $0.5g_{max}$ is attained. The sigmoid function describes the response of a neuron's soma to a given membrane potential. To understand this sigmoid shape, conceptually it can be seen as the integral of a normal distribution. When considering a neural population there are numerous neurons which can be described by different sigmoid functions. The resulting sigmoid shape is a summation of the expected number of neurons that will fire given a specific membrane potential. %%%%%%%%%%%%%%%%%%%%%%%%%
The second function is an average firing rate to population membrane potential integration kernel \begin{align} %%%%%%%%%%%%%%%%%%%%%%%%%%%%%%%%%%%%%%%%%
\label{eqn: Convert}
v_{b}(t) &= G_{b}(t)h_{b}(t)*g(v(t))\\
\label{eqn: Kernel} 
h_{b}(t) &= \begin{cases} 
\frac{1}{\tau_{b}}t\exp\left(-\frac{t}{\tau_{b}}\right) & t \geq 0\\
0 & t <0
\end{cases}. \end{align} Here, $v_{b}(t)$ is the aggregate membrane potential, $h_{b}(t)$ is a kernel that converts firing rates to membrane potentials, $G_{b}(t)$ is the synaptic gain, and $\tau_{b}$ is the time constant. The operator $*$ represents a convolution. The function $h_{b}(t)$ is a time delayed exponential decay function (Figure~\ref{fig: FR2PSP_final}). The subscript $b$ is used to indicate that each neural population is described with a different synaptic gain and time constant. The synaptic gains are time dependant as they are the parameters that are altered to simulate different EEG characteristics (figure~\ref{fig: SeizureSim}). In this description of the Wendling model, numerous model parameters are considered to be stationary: the maximum firing rate~($g_{max}$), threshold voltage~($v_{0}$), sigmoid gradient ($r$), and time constants~($\tau_{b}$). 
%The assumption that the time constants are stationary is similar to assuming that neurons can be modeled as current based and not conductance based.

To solve the convolution in Equation~(\ref{eqn: Convert}), both $g(v(t))$ and $h_{m}(t)$ are transformed into the Laplace domain \begin{align}%%%%%%%%%%%%%%%%%
\label{eqn: Laplace}
G(V(s)) &= \frac{g_{\mathrm{max}}}{1+\exp(r(V(s)-v_{0}))}\\
H_{b}(s) &= \frac{\mathrm{d}}{\mathrm{d}s}\left(-\frac{G_{b}(s)}{\tau_{b}}\left(\frac{1}{s+\frac{1}{\tau_{b}}}\right)\right)\\
H_{b}(s) &= \frac{G_{b}(s)}{\tau_{b}}\left(\frac{1}{s+\frac{1}{\tau_{b}}}\right)^2.
\end{align} It is clear here that the sigmoid function has no functional dependence on time; therefore, the output of the sigmoid will be assumed to be an input firing rate~($u_{b}(t)$) where the time dependence is caused by the time varying membrane potential. The parameters $G_b(t)$ have an unknown functional dependence on time and are, therefore, assumed to be constant for this derivation. This simplifies the convolution equation to \begin{align}%%%%%%%%%%%%%%%
G(V(s)) &= U_{b}(s)\\
H_{b}(s) &= \frac{G_{b}}{\tau_{b}}\left(\frac{1}{s+\frac{1}{\tau_{b}}}\right)^2\\
\label{eqn: LaplaceNMM}
V_{b}(s) &= \frac{G_{b}U_{b}(s)}{\tau_{b}}\left(\frac{1}{s+\frac{1}{\tau_{b}}}\right)^2.
\end{align} Using Equation~(\ref{eqn: LaplaceNMM}), the membrane potential $v_{b}(t)$ can be described as a second order differential equation as follows \begin{align}%%%%%%%%%%%%%%%%%%%%%%%%%%
V_{b}(s) &= \frac{G_{b}U_{b}(s)}{\tau_{b}}\left(\frac{1}{s^2+\frac{2s}{\tau_{b}}+\frac{1}{\tau^2_{b}}}\right)\\
\label{eqn: LaplaceDiff}
V_{b}(s)\left(s^2+\frac{2s}{\tau_{b}}+\frac{1}{\tau^2_{b}}\right) &= \frac{G_{b}(s)U_{b}(s)}{\tau_{b}}.
\end{align} Taking the inverse Laplace transform of Equation~(\ref{eqn: LaplaceDiff}) results in \begin{align} %%%%%%%%%%%%%%%%%%%%%%%%%%%%%%%%
\frac{\mathrm{d}}{\mathrm{d}t^2}v_{b}(t) + \frac{\mathrm{d}}{\mathrm{d}t}\left(\frac{2*v_{b}(t)}{\tau_{b}}\right) + \frac{v_{b}(t)}{\tau^2_{b}} &= \frac{G_{b}(t)u_{b}(t)}{\tau_{b}}\\
\label{eqn: DiffNMM}
\frac{\mathrm{d}}{\mathrm{d}t^2}v_{b}(t) &= \frac{G_{b}(t)u_{b}(t)}{\tau_{b}} - \frac{\mathrm{d}}{\mathrm{d}t}\left(\frac{2*v_{b}(t)}{\tau^2_{b}}\right) -\frac{v_{b}(t)}{\tau^2_{b}}. \end{align} A dummy  variable, $z(t)$, is defined such that Equation~(\ref{eqn: DiffNMM}) can be described as two differential equations, \begin{align} %%%%%%%%%%%%%%%%%%%%%%%%%%%
\label{eqn: dummy1}
\frac{\mathrm{d}}{\mathrm{d}t}v_{b}(t) &= \dot{v}_{b}(t)\\
\label{eqn: dummy2}
\frac{\mathrm{d}}{\mathrm{d}t^2}v_{b}(t) &= \dot{z}_{b}(t)\\
\label{eqn: FR2PSP1}
\dot{v}_{b}(t)&= z_{b}(t)\\
\label{eqn: FR2PSP2}
\dot{z}_{b}(t)&=\frac{G_{b}(t)}{\tau_{b}}n_{b}u_{b}(t)-2\frac{z_{b}(t)}{\tau_{b}}-\frac{v_{b}(t)}{\tau_{b}^{2}}.
\end{align} Here $v_{b}(t)$ is the average membrane potential and $z_{b}(t)$ is its derivative. $G_{b}(t)$ and $\tau_{b}$ are the specific neural populations synaptic gain and time constant. Lastly, $u_{b}(t)$ is the firing rate to the specific neural population considered and $n_{b}$ is a constant used to describe connectivity. The term $n_{b}$ is a consequence of the model simplification, and its derivation is demonstrated in appendix~\ref{sec: AppendixA}.

\begin{figure}%%%%%%%%%%%%%%%%%%%%%%%%%%%%%%%%%%%%%%%%%%%%%%%%%
	\centering
		\includegraphics{FR2PSP.pdf}
	\caption{Impulse response of the firing rate to aggregate membrane potential function. The time constants and synaptic gains used for this figure correspond to that of background EEG~\citep{wendling2002epileptic}. The peaks of each response occur at time $\tau_{b}$ and the maximum membrane potential is $\exp(-1)G_{b}$, where $b$ is replaced by p, s and f for excitatory, and slow and fast inhibitory time constants and gains, respectively. The inhibitory populations response is shown as negative, as this is its net effect on the system.}
	\label{fig: FR2PSP_final}
\end{figure} %
\begin{figure}
\centering
\includegraphics[width = 0.8\textwidth]{BiologicalSimplified.pdf}
\caption{Simplified graphical description of the Wendling model.}
\label{fig: BiologicalSimplified}
\end{figure}

\red{Full mathematical description of the model}
Using Equations~(\ref{eqn: FR2PSP1})-(\ref{eqn: FR2PSP2}) and observing that each synapse of the model (totaling 8), shown in figure~\ref{fig: Biological} requires two differential equations, the number of equations expected would be sixteen. However, this model can be simplified, see Appendix~\ref{sec: AppendixA},~(Figure~\ref{fig: BiologicalSimplified}) to a set of eight stochastic differential equations: \begin{align}%%%%%%%%%%%%%%%%%%%%%%%%%%%%%%%%%%%%%%%
\mathrm{d}v_{\mathrm{p0}}(t)&= z_{\mathrm{p0}}(t)\mathrm{d}t\\
\mathrm{d}z_{\mathrm{p0}}(t)&=\left(\frac{G_{\mathrm{p}}(t)}{\tau_{\mathrm{p}}}n_{\mathrm{p}}g(v_{\mathrm{p}}(t))-2\frac{z_{\mathrm{p0}}(t)}{\tau_{\mathrm{p}}}-\frac{v_{\mathrm{p0}}(t)}{\tau_{\mathrm{p}}^{2}}\right)\mathrm{d}t\\
\mathrm{d}v_{\mathrm{p1}}(t)&= z_{\mathrm{p1}}(t)\mathrm{d}t\\
\label{eqn: Wiener}
\mathrm{d}z_{\mathrm{p1}}(t)&=\left(\frac{G_{\mathrm{e}}(t)}{\tau_{\mathrm{e}}}(\mu +n_{\mathrm{e}}g(v_{\mathrm{e}}(t))-2\frac{z_{\mathrm{p1}}(t)}{\tau_{\mathrm{e}}}-\frac{v_{\mathrm{p1}}(t)}{\tau_{\mathrm{e}}^{2}}\right)\mathrm{d}t + \frac{G_{\mathrm{e}}(t)}{\tau_{\mathrm{e}}}\epsilon(t)\mathrm{d}W\\
\mathrm{d}v_{\mathrm{p2}}(t)&= z_{\mathrm{p2}}(t)\mathrm{d}t\\
\mathrm{d}z_{\mathrm{p2}}(t)&=\left(\frac{G_{\mathrm{s}}(t)}{\tau_{\mathrm{s}}}n_{\mathrm{s}}g(v_{\mathrm{s}}(t))-2\frac{z_{\mathrm{p2}}(t)}{\tau_{\mathrm{s}}}-\frac{v_{\mathrm{p2}}(t)}{\tau_{\mathrm{s}}^{2}}\right)\mathrm{d}t\\
\mathrm{d}v_{\mathrm{p3}}(t)&= z_{\mathrm{p3}}(t)\mathrm{d}t\\
\mathrm{d}z_{\mathrm{p3}}(t)&=\left(\frac{G_{\mathrm{f}}(t)}{\tau_{\mathrm{f}}}n_{\mathrm{f}}g(v_{\mathrm{f}}(t))-2\frac{z_{\mathrm{p3}}(t)}{\tau_{\mathrm{f}}}-\frac{v_{\mathrm{p3}}(t)}{\tau_{\mathrm{f}}^{2}}\right)\mathrm{d}t.
\end{align} In these equations $dW$ represents a Wiener process \iref and is required as $\epsilon(t)\sim N(0,\sigma)$, where $\sigma$ and $\mu$~(Eq.~(\ref{eqn: Wiener})) describe the mean and variance of the stochastic model input, respectively. Further, $v_{\mathrm{p0-3}}(t)$ represent the membrane potential produced by a specific populations synapse and $z_{\mathrm{p0-3}}(t)$ their derivatives. The inputs to each neural population are specified by $v_{b}(t) $, and are the membrane potential of the specific population, where $b$ takes the values of p, e, s and f representing pyramidal, excitatory, and slow and fast inhibitory populations, respectively. Therefore $v_{\mathrm{p}}(t) $ is the output of the model. All $v_{b}(t) $ can be described in terms of $v_{\mathrm{p0-3}}(t)$ as follows \begin{align}%%%%%%%%%%%%%%%%%%%%%%%%%%%%%%%%%%%%%%%%%%%%%%%%%%%%%%%%%%%%%%%%
v_{\mathrm{p}}(t) &= v_{\mathrm{p1}}(t)-c_{4}v_{\mathrm{p2}}(t)-v_{\mathrm{p3}}(t)\\
v_{\mathrm{e}}(t) &= c_{1}v_{\mathrm{p0}}(t)\\
v_{\mathrm{s}}(t) &= c_{3}v_{\mathrm{p0}}(t)\\
v_{\mathrm{f}}(t) &= c_{5}v_{\mathrm{p0}}(t)-c_{6}v_{\mathrm{p2}}(t),
\end{align} where $c_{1}$, $c_{3}$ and $c_{5}$ represent the connectivity strength from pyramidal to excitatory, slow inhibitory and fast inhibitory populations, respectively. The last two connectivity terms, $c_{4}$ and $c_{6}$, represent the connectivity strength from the slow inhibitory to the excitatory and fast inhibitory populations, respectively. Finally, all $n_{b}$ can be defined as connectivity constants:\begin{align}%%%%%%%%%%%%%%%%%%%%%%%%%%%%%%%%%%%%%%%%%%%
n_{p} &=1\\
n_{\mathrm{e}} &=c_{2}\\
n_{\mathrm{s}} &=1\\
n_{\mathrm{f}} &=c_{7},
\end{align} where $c_{2}$ and $c_7$ represent the connectivity strength from excitatory and fast inhibitory populations to the pyramidal population.
\begin{figure}%%%%%%%%%%%%%%%%%%%%%%%%%%%%%%%%%%%%%%%%%%%%%%%%%%%%%%%%%%%%%
	\centering
	    \begin{tabular}{p{0.05cm} l}
    \imagetop{(A)} & \imagetop{\includegraphics[width=0.8\textwidth]{InsetF1.pdf}} \\
    \imagetop{(B)} & \imagetop{\includegraphics[width=0.8\textwidth]{Single_iEEG_Trace_TTN.pdf}}
    \end{tabular}
%\subfigure[]	{\includegraphics{pdf/InsetF1.pdf}
%}\\
%\subfigure[]{	\includegraphics[width =1\textwidth]{pdf/Single_iEEG_Trace_TTN.pdf}
%}\\
	\caption{Simulated seizure using the Wendling model compared to a seizure recorded from an \textsl{in vivo} model of epilepsy.. (a) Here four different sets of synaptic gains are used to simulate the seizure. Four types of neural activity are demonstrated in this figure: background EEG (1-10s), interictal (11-20s), low voltage high frequency (21-30s) and seizure (31-40s). The last ten seconds shows background data again. Four insets are demonstrated in the figure each corresponding to a different type of activity. The insets in order from left to right show: background, interictal, low voltage high frequency and seizure activity. (b) Tetanus toxin focal seizure data, where four types of activity are demonstrated: background(1-5s), interictal (6-20s), low voltage high frequency (21-35s) and seizure (36-60s).}
	\label{fig: SeizureSim}
\end{figure}%%%%%%%%%%%%%%%%%%%%%%%%%%%%%%%%%%%%%%%%%%%%%%%%%%%%%%%%%%%%%%%%%%%%%%


\red{Simulation of model}
This set of continuous stochastic differential equations is discretised using Euler-Mariyama's method \iref, to simulate EEG \begin{align}
\label{eqn: EulerW1}
v_{\mathrm{p0},k+1}&=v_{\mathrm{p0},k}+Tz_{\mathrm{p0},k}\\
z_{\mathrm{p0},k+1}&=z_{\mathrm{p0},k}+T\left(\frac{G_{p,k}}{\tau_{p}}n_{p}g(v_{p,k})-2\frac{z_{\mathrm{p0},k}}{\tau_{p}}-\frac{v_{\mathrm{p0},k}}{\tau_{p}^{2}}\right)\\
v_{\mathrm{p1},k+1}&=v_{\mathrm{p1},k}+Tz_{\mathrm{p1},k}\\
z_{\mathrm{p1},k+1}&=z_{\mathrm{p1},k}+T\left(\frac{G_{\mathrm{e},k}}{\tau_{\mathrm{e}}}(\mu +n_{\mathrm{e}}g(v_{\mathrm{e},k})-2\frac{z_{\mathrm{p1},k}}{\tau_{\mathrm{e}}}-\frac{v_{\mathrm{p1},k}}{\tau_{\mathrm{e}}^{2}}\right) + \sqrt{t}\frac{G_{\mathrm{e},k}}{\tau_{\mathrm{e}}}\epsilon_{t}\\
v_{\mathrm{p2},k+1}&=v_{\mathrm{p2},k}+Tz_{\mathrm{p2},k}\\
z_{\mathrm{p2},k+1}&=z_{\mathrm{p2},k}+T\left(\frac{G_{\mathrm{s},k}}{\tau_{\mathrm{s}}}n_{\mathrm{s}}g(v_{\mathrm{s},k})-2\frac{z_{\mathrm{p2},k}}{\tau_{\mathrm{s}}}-\frac{v_{\mathrm{p2},k}}{\tau_{\mathrm{s}}^{2}}\right)\\
v_{\mathrm{p3},k+1}&=v_{\mathrm{p3},k}+Tz_{\mathrm{p3},k}\\
\label{eqn: EulerW8}
z_{\mathrm{p3},k+1}&=z_{\mathrm{p3},k}+T\left(\frac{G_{\mathrm{f},k}}{\tau_{\mathrm{f}}}n_{\mathrm{f}}g(v_{\mathrm{f},k})-2\frac{z_{\mathrm{p3},k}}{\tau_{\mathrm{f}}}-\frac{v_{\mathrm{p3},k}}{\tau_{\mathrm{f}}^{2}}\right),
\end{align} where $k$ represents the current sample and $T$ is the period between them. The static parameter values are shown in Table~\ref{tab: Static}. The variance of the input,$\sigma$, is specified such that 99.7\% of realisations drawn from the Gaussian distribution fall within the specified maximum and minimum firing rate. For the cases where the realisations from the Gaussian distribution are not contained within the limits specified, the specific sample of interest is redrawn from the same Gaussian distribution until the firing rate falls within the specified range. In Table~\ref{tab: Static}, the parameters $G_{\mathrm{p},k}$, $G_{\mathrm{e},k}$, $G_{\mathrm{s},k}$ and $G_{\mathrm{f},k}$ are not specified as these parameters will vary for different simulations. However, for this simulation, it is assumed that \begin{align}
G_{\mathrm{p},k} = G_{\mathrm{e},k}.
\end{align} This assumption can be made as the excitatory population in this model specifies recurrent connections between pyramidal neurons.  
\singlespacing
\small
\begin{center}%%%%%%%%%%%%%%%%%%%%%%%%%%%%%%%%%%%%%%%%%%
	\begin{table}
			\caption{Static Model Parameters~\citep{wendling2002epileptic}. Here p, e, s and f represent populations of pyramidal neurons and excitatory, and slow and fast inhibitory interneurons, respectively.}
		\begin{tabular}{||p{4cm}|p{6cm}|p{1.5cm}|p{1.2cm}||}\hline
			 \textsc{Model parameter}  & \textsc{Physical description} & \textsc{Value} & \textsc{Units}  \\\hline\hline
			 $\tau_{p}$ & Time constant for pyramidal neurons & 100 & $s^{-1}$\\\hline
			 $\tau_{\mathrm{e}}$ & Time constant for excitatory neurons & 100 & $s^{-1}$\\\hline
			 $\tau_{\mathrm{s}}$ & Time constant for slow inhibitory neurons & 35 & $s^{-1}$\\\hline
			 $\tau_{\mathrm{f}}$ & Time constant for fast inhibitory neurons & 500 & $s^{-1}$\\\hline
			 $c$ & Connectivity constant & 135 & NA\\\hline
			 $c_{1}$ & Connectivity constant (p - e) & $c$ & NA \\\hline
			 $c_{2}$ & Connectivity constant (e - p) & $0.8c$ & NA\\\hline
			 $c_{3}$ & Connectivity constant (p - s) & $0.25c$ & NA \\\hline
			 $c_{4}$ & Connectivity constant (s - p)& $0.25c$ & NA\\\hline
			 $c_{5}$ & Connectivity constant (p - f) & $0.3c$ & NA\\\hline
			 $c_{6}$ & Connectivity constant (s - f) & $0.1c$ & NA\\\hline
			 $c_{7}$ & Connectivity constant (f - p) & $0.8c$ & NA\\\hline
			 $g_{max}$ & Maximum firing rate & 5 & Hz \\\hline
			 $v_{0}$ & PSP for which 50\% firing rate is achieved & 6 & $mV^{-1}$\\\hline
			 $r$ & Gradient of sigmoid function & 0.56 & NA \\\hline
			 $f_{max}$ & Maximum input firing rate & 150 & Hz \\\hline
			 $f_{min}$ & Minimum input firing rate & 30 & Hz \\\hline
			 $\mu$ & Input mean firing rate & 90 & $Hz$\\\hline
			 $\sigma$ & Variance of input firing rate & 15 & $Hz$\\\hline\hline 
		\end{tabular}
		\label{tab: Static}
	\end{table}
\end{center}%%%%%%%%%%%%%%%%%%%%%%%%%%%%%%%%%%%%%%%%%%%%%%%%%%%%%%%%%%
\onehalfspacing

\subsection{Estimation}

\red{Generic description on a nonlinear system}
A generic system is defined where \begin{align}
\label{eqn: NonlinEstS}
\mathbf{\dot{x}}(t) &= \mathbf{A}(\mathbf{x}(t),\mathbf{\theta}(t)) + \mathbf{B}(\mathbf{u}(t)) + \mathbf{n}(t)\\
\label{eqn: NonlinEstO}
\mathbf{y}(t)  &= \mathbf{C}(\mathbf{x}(t)) +\mathbf{D}(\mathbf{u}(t))+\mathbf{r}(t),
\end{align} where boldface indicates a matrix or vector. Here, $\mathbf{x}(t)$ is the state vector and $\dot{\mathbf{x}}(t)$ is its derivative, where
\[ \mathbf{\dot{x}}(t) = \left[ \begin{array}[pos]{c}
\dot{x}_{1}(t)\\
\vdots \\
\dot{x}_{n}(t) \end{array} \right] .\] $\mathbf{A}$ and $\mathbf{B}$ are the state and input functions, respectively and $\mathbf{u}(t)$ is the input to the model. $\mathbf{C}$ and $\mathbf{D}$ are the output and input-to-output functions, respectively. The output of the model is $\mathbf{y}(t)$ with $\mathbf{n}(t)$ the model uncertainty, and $\mathbf{r}(t)$ the observation noise. Here, $\mathbf{n}(t)$ takes into account that the model is not a perfect descriptor of the modeled system as well as accounting for unknown inputs to the system, and $\mathbf{r}(t)$ describes the maximum amplitude of the noise in the observations. Both $\mathbf{n}(t)$ and $\mathbf{r}(t)$ are zero mean Gaussian distributed with a system dependant variance. The assumption of Gaussian distribution is only valid when the number of samples for the estimation procedure is large (central limit theorem \iref). In particular, the sampling rate needs to be much greater than the maximum frequency that the model dynamics describes. This is often referred to as an oversampled system \iref.

\red{Introduction to the UKF}
Estimation algorithms usually estimate the model states, $\mathbf{x}(t)$, given some observation, $\mathbf{y}(t)$ . One method that is often used for estimating linear systems is the Kalman filter. The Kalman filter consists of two steps: prediction and correction. In the prediction step, model states are propagated through the system and are used to determine the expected value of the states at the next time step. Using a first order Euler-Maruyama method, this can be described by \begin{align}
\label{eqn: StateProgL}
\mathbf{x}_{k+1}^{-} &= \mathbf{x}_{k} + T(\mathbf{A}\mathbf{x}_{k} +\mathbf{B}\mathbf{u}_{k})+\sqrt{T}\mathbf{n}_{k}\\
\label{eqn: YProp}
\mathbf{y}_{k+1}^{-}  &= \mathbf{y}_{k} + T(\mathbf{C}\mathbf{y}_{k}+\mathbf{D}\mathbf{u}_{k}) +\sqrt{T}\mathbf{r}_{k},
\end{align} where the subscript in $\mathbf{x}_{k}^{-}$ is used to indicate the current sample and $T$ is the sampling period. The stochastic variables are $\mathbf{n}_{k}\sim N(0,\mathbf{\sigma}_{n})$ and $\mathbf{r}_{k}\sim N(0,\mathbf{\sigma}_{r})$, where $\mathbf{\sigma}_{n}$ and $\mathbf{\sigma}_r$ are the standard deviations for each respective noise process and $N(\cdot)$ is a Gaussian distribution. The superscript in $\mathbf{x}_{k}^{-}$ is used to indicate that this estimate is a prediction that has not yet been corrected by the current observation. Performing this kind of prediction for a nonlinear system would be inaccurate as the propagation of states like this in a system would require the assumption that the maximum error in the states remains constant for all time. However, in a nonlinear system this is not true, as a state's error can change from one prediction to the next. In particular, if the original state estimate is incorrect in a nonlinear system it is possible that the error can increase in the next iteration of the estimation procedure. In order to account for this error, or the altering of state covariance, an unscented filter is used in the prediction step for nonlinear systems. The advantage of an unscented filter over local linearisation techniques is that speed is improved and discontinuities can be handled. 

\red{The unscented filter}
The unscented filter is completely described by a state's mean and covariance such that:\begin{align}
\label{eqn: Unscented_Transform1}
\mathbf{\mathcal{X}}_{n} &= \mathbf{\overline{x}}_{k} + (\sqrt{\kappa+D_{x}\mathbf{P_{xx,k}}})_{n} \quad n=1,\hdots,D_x\\
\label{eqn: Unscented_Transform2}
\mathbf{\mathcal{X}}_{n+D_{x}} &= \mathbf{\overline{x}}_{k} - (\sqrt{\kappa+D_{x}\mathbf{P_{xx,k}}})_{n} \quad n=1,\hdots,D_x,\\
\label{eqn: Unscented_TransformY1}
\mathbf{\mathcal{Y}}_{n} &= \mathbf{\overline{y}}_{k} + (\sqrt{\kappa+D_{y}\mathbf{P_{yy,k}}})_{n} \quad n=1,\hdots,D_y\\
\label{eqn: Unscented_TransformY2}
\mathbf{\mathcal{Y}}_{n+D_{y}} &= \mathbf{\overline{y}}_{k} - (\sqrt{\kappa+D_{y}\mathbf{P_{yy,k}}})_{n} \quad n=1,\hdots,D_y,
\end{align} where $\mathbf{\overline{x}}_{k}$ and $\mathbf{\overline{y}}_{k}$ are the current state estimate and observation. $(\sqrt{\cdot})_{n}$ denotes the $n$th row of the matrix square root. $D_{x}$ and $D_{y}$ indicate the number of states in the system and the number of observations. Covariance matrices, $\mathbf{P_{xx,k}}$ and $\mathbf{P_{yy,k}}$, are the expected error of the current state and observation. The points $\mathbf{\mathcal{X}}$ are called sigma points and represent the states one standard deviation away for the estimated mean. The term $\kappa$ is a predefined constant, which determines the relative effect of the propagation of the mean. If $\kappa$ is equal to zero then the system mean is not propagated as a sigma point. However, if $\kappa$ is greater than zero then \begin{align}
\mathbf{\mathcal{X}}_{0} &= \mathbf{\overline{x}}_{k}.
\end{align} Therefore, 2$D_{x}$ sigma points are assigned when $\kappa$ is zero and 2$D_{x}$+1 sigma points are assigned when it is greater than zero. The sigma points are propagated through the system in order to update the expectation about the state mean and error: \begin{align}%%%%%%%%%%%%%%%%%%%%%%%%%%%%%%%%%%%%
\mathbf{\mathcal{X}}_{n,k+1} &= \mathbf{\mathcal{X}}_{n,k}+ T(\mathbf{A}(\mathbf{\mathcal{X}}_{n,k}) +\mathbf{B}(\mathbf{u}_{k})) +\sqrt{T}{n}_{k}\\
\overline{\mathbf{x}}_{k+1}^{-} &= \frac{1}{2D_{x}+\kappa}\sum_{n=1}^{2D_{x}} \mathbf{\mathcal{X}}_{n,k+1}\\
\mathbf{P}_{xx,k+1}^{-} &= \frac{1}{2D_{x}+\kappa}\sum_{n=1}^{2D_{x}} (\mathbf{\mathcal{X}}_{n,k+1} -\mathbf{\overline{x}}_{k+1}^{-})(\mathbf{\mathcal{X}}_{n,k+1}-\mathbf{\overline{x}}_{k+1}^{-})^{\top} + \mathbf{Q}.%%%%%%%%%%%%%%%%%%%%%%%%%%%%%%%%%%%%
\end{align} $\overline{\mathbf{x}}_{k+1}^{-}$ and $\mathbf{P}_{xx,k+1}^{-}$ are the predictions for the state and state covariance matrices. The negative superscript is used to indicate an uncorrected prediction. The term $\mathbf{Q}$ is the expectation of the model error $n_{k}$ and $(\cdot)^{\top}$ indicates the transpose. It is now possible to make a prediction about the observation at sample $k+1$ by propagating the sigma points through Equation~(\ref{eqn: YProp}) \begin{align} %%%%%%%%%%%%%%%%%%%%%%%%%%%%
\mathbf{\mathcal{Y}}_{n,k+1} &= \mathbf{\mathcal{Y}}_{n,k} + T(\mathbf{C}(\mathbf{\mathcal{X}}_{n,k+1})+ \mathbf{D}(\mathbf{u}_{k}))+ \sqrt{T}\mathbf{r}_{k}\\
\overline{\mathbf{y}}_{k+1}^{-} &= \frac{1}{2D_{x}+\kappa}\sum_{n=1}^{2D_{x}} \mathbf{\mathcal{Y}}_{n,k+1}\\
\label{eqn: statecovg}
\mathbf{P}_{xy,k+1}^{-} &= \frac{1}{2D_{x}+\kappa}\sum_{n=1}^{2D_{x}} (\mathbf{\mathcal{X}}_{n,k+1}-\overline{\mathbf{x}}_{n,k+1}) (\mathbf{\mathcal{Y}}_{n,k+1}-\overline{\mathbf{y}}_{k+1}^{-})^{\top}\\
\mathbf{P}_{yy,k+1}^{-} &= \frac{1}{2D_{x}+\kappa}\sum_{n=1}^{2D_{x}} (\mathbf{\mathcal{Y}}_{n,k+1}-\overline{\mathbf{y}}_{k+1}^{-}) (\mathbf{\mathcal{Y}}_{n,k+1}-\overline{\mathbf{y}}_{k+1}^{-})^{\top} +\mathbf{R},%%%%%%%%%%%%%%%%%%%%%%%%%%%%%%%%%%%%%%%%%%%%
\end{align} where $\overline{\mathbf{y}}_{k+1}^{-}$ and $\mathbf{P}_{yy,k+1}^{-}$ are the predictions for the model output and its covariance, respectively. $\mathbf{P}_{xy,k+1}^{-}$ is the covariance matrix of the states and observations and $\mathbf{R}$ is the expectation of the observation error $\mathbf{r}_{k}$.

\red{How states are predicted using the unscented transform}
The predictions of the states,$\overline{\mathbf{x}}_{k+1}^{-}$, and observations, $\overline{\mathbf{y}}_{k+1}^{-}$, now need to be corrected based on the observations. This is achieved by determining the Kalman gain and updating the predictions based on the current observation: \begin{align}
\mathbf{K} &= \mathbf{P}_{xy,k+1}^{-}(\mathbf{P}_{yy,k+1}^{-})^{-1}\\
\overline{\mathbf{x}}_{k+1} &= \overline{\mathbf{x}}_{k+1}^{-} + \mathbf{K}(\mathbf{y}_{k+1}-\overline{\mathbf{y}}_{k+1}^{-})\\
\mathbf{P}_{xx,k+1} &= \mathbf{P}_{xx,k+1}^{-} - \mathbf{K}(\mathbf{P}_{xy,k+1}^{-})^{\top},
\end{align} where $\mathbf{y}_{k}$ is the observation, $\overline{\mathbf{x}}_{k+1}$ is the corrected estimate of the state and $\mathbf{P}_{xx,k+1}$ is the estimate of its error. This set of equations describes the UKF and how it can be used to estimate states. However, for this study, estimation of states and parameters is required (dual estimation).

\red{Definition of slow state matrix and its dynamics}
The parameters that are being estimated are $G_{\mathrm{p},k}$, $G_{\mathrm{e},k}$, $G_{\mathrm{s},k}$ and $G_{\mathrm{f},k}$, where it is assumed that \begin{align}
G_{\mathrm{p},k} = G_{\mathrm{e},k}.
\end{align} Therefore, three parameters need to be estimated, and a parameter matrix is defined
\[ \mathbf{\theta}_{k} = \left[ \begin{array}[pos]{c}
G_{\mathrm{p},k}\\
G_{\mathrm{s},k} \\
G_{\mathrm{f},k} \end{array} \right] .\] The original state matrix is then augmented with the parameter matrix 
\[ \mathbf{x}_{k} = \left[ \begin{array}[pos]{c}
\mathbf{x}_{k}\\
\mathbf{\theta}_{k} \end{array} \right] .\] The change in model parameters occurs at a longer time scale than the model states. Therefore, for convenience, the model states and parameters are referred to as fast and slow states. The next issue to consider is the description of the dynamics for the slow states. Due to the prediction correction steps of the unscented Kalman filter these slow states can be assigned trivial dynamics such that
\begin{align}
\label{eqn: parameterdynamics}
\mathbf{\theta}_{k+1} &= \mathbf{\theta}_{k} + \mathbf{\eta_{k}}\\
E(\mathbf{\theta}_{k+1}) &= \mathbf{\theta}_{k}\\
P_{\mathbf{\theta} \mathbf{\theta},k+1} &= \mathbf{\Psi},
\end{align} where $E(\cdot)$ is the expectation function and $\mathbf{\eta}_{k} \sim N(0,\mathbf{\sigma})$.

\red{Intialisation of UKF for stationary parameters}
When initialising the unscented Kalman filter the model uncertainty and initialisation standard deviation for each state needs to be specified. To determine the accuracy of the model when estimating fast states, initial estimations are performed under the assumption that the model's slow states are stationary. Assuming that the model slow states are stationary allows for the uncertainty in these states to remain low. This uncertainty characterises possible model inaccuracy, and also allows the slow state to slowly vary until they converges to their true values. Standard deviations in the estimation description are described such that one standard deviation from the midpoint of the specified slow states range encompasses all possible values for the particular state. Further, the uncertainty of states directly affected by the stochastic model input is increased to account for the unpredictable nature of this signal when simulating. This is required as a stochastic process cannot be estimated accurately. 

Due to the stochastic model input, it is non-trivial to determine the maximum limit of the physiological range describing the fast states. Therefore, numerous simulations are performed and the resulting mean and standard deviation for each state is used as the initial mean and standard deviation for estimation: \begin{align}
v_{b,0} = \frac{1}{n}\sum\limits_{i=1}^n E(v_{i,b,k}), \end{align} where $n$ indicates the number of simulations performed and $v_{b,0}$ and $E(v_{b,k,i})$ indicate the expected value of the initialised fast states and the expected value of each simulation, respectively. For the standard deviation, the average over multiple simulations for a normal distribution can be determined by \begin{align}
\sigma^2_{b,0} = \frac{1}{n-1}\sum\limits_{i=1}^n \sigma^2_{b,k},\end{align} where $\sigma^2_{b,0}$ and $\sigma^2_{b,k}$ are the variances of the initial guess and of each state's simulation results, respectively.


Next the assignment of the mean and covariance of the slow states is considered. For this calculation it is assumed that all possible values of the slow states are equally probable. The range of these slow states is infinite; however, there is a physiological bound on their values~\citep{wendling2002epileptic}. These bounds will be used and are defined as $\theta_{b,max}$ and $\theta_{b,min}$. The mean and covariance of the slow states are\begin{align}
\label{eqn: InitThetaMean}
\overline{\theta}_{b,0} = \frac{\theta_{b,max}+\theta_{b,min}}{2}\\
\label{eqn: InitThetaCoV}
P_{\theta\theta,0} = (\overline{\theta}_{b,0}-\theta_{b,min})^2.
\end{align}

\red{Intialisation of UKF for varying parameters}
When estimating slow states that vary within a single simulation, the uncertainty assigned to these parameters is increased. This increase in uncertainty guarantees that the estimation procedure will not converge to a specific slow state and remain there, but instead will track it as it varies. 

\red{Estimation of model input mean}
Finally, estimation of the stochastic input's mean to the neural mass is considered. Here it is assumed that the model input mean is varying slowly. By doing so the input mean can be augmented to the state matrix and assigned trivial dynamics. With the physiological bounds on the input specific by~\cite{wendling2002epileptic}, the initial mean and standard deviation of the input can be defined by equations~\ref{eqn: InitThetaMean}-\ref{eqn: InitThetaCoV}.

\red{Robustness test}
The performance of the estimation procedure is determined. Initially, only estimation of fast states is considered. This is then developed to the full estimation procedure where all model parameters and the input mean are estimated. The estimation procedure is then tested under numerous observation noise conditions, and with varying levels of error in the initialisation of states.



\section{Results}

The results section is split into three components: model selection, estimation, and estimation validation. In the section model selection we demonstrate that the neural mass model can mimic numerous dynamics observed in hippocampal EEG. For the estimation results we show that the unscented Kalman filter can be used to estimate the synaptic gains and input mean of a neural mass model with EEG observations. In estimation validation we then explore how descriptive the model estimates are of the recorded EEG. This is achieved by simulating data using the estimated synaptic gains and input mean. The simulated data is then compared to the recorded EEG in order to determine whether the estimates are capable of replicating the key characteristics observed.

\subsection{Model Selection}

The neural mass model of the hippocampus has been shown to be capable of replicating the frequency dynamics of hippocampal EEG~\citep{wendling2002epileptic}. In Figure~\ref{fig: EEG} two EEG traces are demonstrated, the first of recorded EEG from an animal that has had tetanus toxin injected into its hippocampus, and the second of simulated data using the neural mass model. The simulated data is generated by altering the synaptic gains and the input mean of the neural mass model of the hippocampus. In both traces the red dashed lines indicate the beginning and end of a seizure. Their is a clear relationship between the amplitude of the simulated data and the recorded EEG. In particular, at seizure onset both amplitudes of the simulated and recorded data increase. Prior to seizure termination their is a marked decrease in amplitude in both the recorded and simulated data. 
Even though their are clear similarities between the two traces, their are also clear differences. These differences are due to the input to the neural mass model, which is stochastic. Since the input is stochastic for the model, and unknown for the recorded EEG their will always be differences between the two. However, using this technique we can capture the dynamics that are important in terms of seizure initiation and termination. In this study, we evaluate what are the most likely set of physiological parameters that describe the transitions between seizure and non-seizure, and how these parameters alter during seizure.

\begin{figure}[ht]
 	\centering
 		\includegraphics{fig/EEG.pdf}
 	\caption{Recorded and Simulated EEG. (\textbf{A}) Recorded EEG from an animal that has had a hippocampal tetanus toxin injection. (\textbf{B}) Simulated data generated from the neural mass model of the hippocampus. The red dotted lines indicate the start and end of a seizure in both traces.}
 	\label{fig: EEG}
 \end{figure}

\subsection{Estimation}

Hippocampal EEG can be mimicked by altering the synaptic gains of the neural mass model. However, a recent study has shown that there are network changes involved in seizure initiation and termination~\citep{truccolo2011single}. Therefore, we also consider the input mean to the modeled neural mass to be time varying. For this study, we have used recordings from four different control animals, and consider four seizures from each of them. All parameters are initialised identically and the same algorithm is used for all recordings. 

The first of five traces in figures \textbf{A}-\textbf{D} show the recorded EEG of seizure one for each animal~(Figure~\ref{fig: EstimationResults}). The corresponding four traces are the results for the estimated model parameters. All seizures start at time zero and end at the dashed line with the same color as the estimation results. At seizure initiation a change in all model parameters occurs. However, the dynamics of these model parameters at seizure initiation and termination are different for each animal, and in some cases vary within animals for different seizures. 

The estimated parameters for animal one~(\textbf{A}) are similar prior to seizure for all parameters except the slow inhibitory synaptic gain. Post seizure the results show that the excitatory and slow inhibitory synaptic gains are similar, but the input mean and fast inhibitory synaptic gain are different between seizures. This is similar to the results from the seizure itself. All parameters increase at seizure initiation, except the slow inhibitory synaptic gain. The results in this figure also demonstrates that at seizure initiation the estimated mean input firing rate alters. This is of particular interest, as to our knowledge, all current estimation techniques assume that this parameter remains constant through all EEG. 

For animal two the preseizure and post seizure estimation results are similar for the excitatory and slow inhibitory gain. However, the estimates for the input mean, and fast inhibitory gain vary from seizure to seizure. All model parameters, other than the slow inhibitory gain, increase at seizure initiation.  

For animals three and four the model parameters also alter at seizure initiation and termination. However, the dynamics of the parameters at this transition varies from seizure to seizure.

When comparing the results between animals there are clear differences between the estimated model parameters. In particular the evolution of seizure varies largely between animals. When considering animal one and two, the overall dynamics of the model parameters at the transition to seizure are similar. However, if we consider for example the excitatory synaptic gains for both animals there are clear differences~(Figure~\ref{fig: SZComp}). In animal one, the excitatory gain increases drastically at seizure initiation and then slowly decreases until seizure termination. For animal two, there is a slow increase in the excitatory gain throughout seizure, and a drastic decrease at seizure termination. This suggests that there are not only differences in the estimated mechanisms between animals, but that even when the mechanisms are similar their evolution through seizure may vary. 

The results from the estimation procedure above suggest that the mechanisms involved in seizure may vary between animals, and possibly within animals over longer time periods. This contradicts the majority of studies that have shown single parameter sets that are capable of describing the transition from background to seizure. The reason for this is two fold: the first is that the majority of studies attempt to match the frequency of the observed EEG to the simulated model, whereas the unscented Kalman filter approach makes use of the information in the amplitude of the signal to determine the most likely parameter values that describe each observation. The second reason is that the unscented Kalman filter makes the assumption that the model is not capable of completely describing the observed EEG (model uncertainty), this uncertainty incorporates model inadequacy as well as the stochastic effect of the input on the model output. In this way the unscented Kalman filter is more robust at dealing with stochastic inputs, and this may lead to varying results between the two methods. 

The Kalman filter is a Markov process, so there is an assumption that all the required information to determine the most likely states at the next time point can be compressed into the mean and covariance of the current states. This may not necessarily be true for a system like the brain. Further, the greater the number of states that are used in the unscented Kalman filter framework the more likely the algorithm is to diverge from its true value. That being said, we have tested the algorithm on noisy simulated data, generated from the model, and have shown that it converges to the true parameter values.

% Compare estimation results when input held constant to esitmation results with varying input

\begin{figure}[ht]
 	\centering
	 \vspace*{-2cm}
 		\includegraphics[max size={\textwidth}{\textheight}]{fig/EstimationFigures1.pdf}
 	\caption{Results from the estimation of four seizures from four different animals . (\textbf{A}), (\textbf{B}), (\textbf{C}), (\textbf{D}) Estimation results for four seizures(SZ) from animal 1-4, respectively. The colors in all graphs correspond to the legend above the figure. The first estimated seizure is shown in the top graph, where the initiation of seizure begins at time 0s and ends at the dotted black line. In the corresponding graphs the estimation results for the excitatory ($\alpha_{\mathrm{p}}$), slow inhibitory ($\alpha_{\mathrm{s}}$) and fast inhibitory ($\alpha_{\mathrm{f}}$) synaptic gains are demonstrated, where all seizures initiate at 0s and terminate at the dotted line with the same color. Lastly, the estimate for the mean input firing rate to the modeled population is demonstrated.}
 	\label{fig: EstimationResults}  
 \end{figure}

To validate the estimation results above we consider seizure one for each animal. Using the estimation results from seizure one we perform a forward simulation (Figure~\ref{fig: SZComp}). The results for seizure one from each animal are shown. It is clear from these figures that the estimation technique is capable of reproducing model parameters that are capable of producing simulated data similar to recorded EEG, in terms of the amplitude.  

\begin{figure}[ht]
 	\centering
 		\includegraphics{fig/AnimalComp.pdf}
 	\caption{Estimation Results for Multiple Animals and Simulated EEG. (\textbf{A}) The results from seizures for four different animals. The estimated physiological parameters from each seizure (SZ) is specified by a different color line. All seizures initiate at time 0s and end at the dotted line that is the same color as the estimation results for the animal. (\textbf{B}) Artificial EEG generated using the neural mass model of the hippocampus. The artificial EEG is generated using the parameters from \textbf{A} with the same color as the traces in \textbf{B}.}
 	\label{fig: SZComp}
 \end{figure}

To demonstrate the robustness of the unscented Kalman filter to the starting values chosen for the estimation we have performed a monte carlo analysis~(Figure~\ref{fig: MonteResults}). In each trace the mean and covariance of the estimation over 100 simulations is shown. From these results it is clear that there is a higher level of variance on the fast synaptic inhibition ($\alpha_f$) and the input mean post seizure. However, the slow inhibitory gain standard deviation is higher preseizure. To compare the standard deviation over 100 estimations with the predicted error from the unscented Kalman filter we have plotted both results. The results show that there are similarities between the estimated error, and the error over multiple different intialised parameters. In particular, the estimation results preseizure for slow inhibition show a high standard deviation for both the single and monte carlo estimation procedure.

The results from figure~\ref{fig: MonteResults} also demonstrate the robustness of the unscented Kalman filter to the initialised parameters. In particular, when considering the mean over the monte carlo simulation and the results from a single estimation, there are clear similarities. The transition to and from seizure for all the model parameters have the same trends, and the standard deviation in both sets of results decreases during seizure periods. This is due to an increase in the singal to noise ration in seizure periods. 

The reason for a high variance on the input is apparent, as is is the mean of a stochastic element. When considered the fast synaptic inhibition, the high variance may be due to the lack of memory in the Kalman filter. With a lack of memory, high frequency activity, which results from fast inhibition in this model, may be overestimated. It is possible that limiting the variance on the parameter may results in better estimates. However, by decreasing the variance of this parameter the time it would take to track would decrease.

\begin{figure}[ht]
	\centering
		\includegraphics{fig/MonteResults.pdf}
	\caption{Comparison of the mean and standard deviation of a single and monte carlo analysis of the estimation technique. \textbf{A} Results from a single estimation are shown, where the standard deviation of each estimate made by the unscented Kalman filter is also shown in red. \textbf{B} The mean and standard deviation of 100 estimations using the same data set with different initial model parameters.}
	\label{fig: MonteResults}
\end{figure}
 


\section{Discussion}

\red{Why estimation}

\red{Effect of stochastic input}

\red{Estimation of model parameters}

\red{Initialisation error}

\red{Observation Noise}

\red{Parameters varying}

% 
% \section{Discussion}

\red{What does this study add}
Epilepsy is a poorly understood disorder, and there is uncertainty about the mechanisms involved in seizure generation and termination. However, with computational models of neural recording becoming more descriptive of physiology, it is becoming possible to gain insights into the underlying mechanism involved in recorded EEG. Computational models of EEG are non-linear, and standard estimation techniques cannot be used to approximate the physiological states from these models. In this paper, the application of the unscented Kalman filter to the Wendling model has been considered to help further the understanding of the mechanisms involved in seizure generation and termination.

\red{Why  neural mass model, and is it relevant to physiology}

Are neural mass models, good models of the brain? To answer this question a definition for the word model is required. In this paper a model is considered to be an approximate descriptor of a system, which in this case is the brain. Since neural mass models are derived from aspects of neural physiology they can be considered to be models of the brain. Whether they are good models of the brain depends on how well they are able to replicate the observations they are intended to mimic. That being said, a good neural mass model is merely one of numerous descriptors of the brain, and without further evaluation of the results obtained from the model with physiological studies a model provides little evidence of casual relationships. However, by using these models insight can be gained into what aspects of physiology should be evaluated, and can provide a method for aspects of physiology that can  not normally be observed, to be crudely estimated. For this study, a good model will be considered to be any model that can describe normal and seizure EEG with distinctly different physiological properties. Further, this model should take into account all relevant physiological studies that have demonstrated causal relationships in the brain that would affect the dynamics produced by the model. 

The relevance of computational models of neural masses to real physiology is still an open question, and we do not propose that the results shown here are indicative of actual physiological mechanisms. However, the derivation of the computational model considered is based on physiological observations made in the hippocampus\citep{freeman1963electrical,wilson1973mathematical,white2000networks}, and the ability of the model to mimic recorded EEG~(Figure~\ref{fig: EEG}) from the hippocampus is promising. If we assume that the the model is a good description of the cortical region we record EEG from, then we can at least say that the estimation results provide some insight into what may be the most likely causes of seizures. If we further assume that variability in the underlying structure of recorded EEG is due to different physiological mechanism at play, we can also conclude that variability in the estimation results between, and within, animals is due to different mechanisms of action, and that seizures are indeed specific to each individual animal.

\red{Why estimation (UKF)}

\red{Variability in animal seizures between and within animals}

\red{UKF tracking of model dynamics}

\red{Variability between seizures, is this just an estimation phenomena from initialised states}

\red{Why is this useful}

\red{Shortcomings}

\red{Why estimation}


\red{Relevance to physiology}

The unscented Kalman filter is capable of tracking changes in the dynamics in recorded EEG. This is clear when considering the estimation results~(Figure~\ref{fig: EstimationResults}) where at seizure initiation and termination there are clear changes in the estimated values and the dynamics of all model parameters. Figure~\ref{fig: SZComp} further demonstrates the estimated model parameters provide a good description of the transition from background to seizure. However, there are clear differences between the recorded and simulated EEG. The first reason for this is that the input to the simulated data varies from simulation to simulation because it is stochastic. Secondly, we assume that the stochastic input to the model has a constant variance, which may not necessarily be the case. In fact by increasing the variance to the simulated model recorded seizure activity is  more accurately replicated; however, pre-seizure and post seizure EEG are less accurately mimicked.

The reliability of the estimates of each model parameter can be seen by considering the pre-seizure results~(Figure~\ref{fig: EstimationResults}). For animal one the pre-seizure estimates for all parameters, except slow inhibition, converge to similar values. This provides good evidence that the estimation is indeed extracting the key characteristics of the recorded EEG. Further, for animal one the estimated model parameters during seizure converge to similar values and have very similar trends. The post seizure results for slow inhibition and excitation are similar between seizures; this is not the case for the input mean and fast inhibition. We observe similar results in all animals, with a few exceptions. For animal two pre-seizure results for all parameters except fast inhibition are similar, there is variability in the values the algorithm converges to during seizures; however, the trends in all estimated parameters for each seizure are alike. Animal 3 has very similar results to animal two in terms on parameters that converge to similar values and the trends in the model parameters. 

The results for animal four are different to the results from the other three animals. There is variability in all the estimated model parameters for all periods shown. The only parameter that converges to similar values between seizures is the excitory synaptic gain pre-seizure. Of particular interest is the estimation results from seizure three, here the model parameters converge to values that are physiologically implausible. There, are two possible reasons for this result: firstly, divergence in the unscented Kalman filter; secondly, model inadequacy when considering this particular seizure. 

The estimated model parameters for each animal are different~(Figure~\ref{fig: SZComp}). This is clear when considering the estimated results for the seizure period. However, the results from pre-seizure and post seizure also show different mechanisms of action. This results may indicate that there are different regulatory mechanisms that occur in the brain in order to suppress seizures, or that there are variations in the networks structure recorded from in the different animals. Another possible reason may be slightly different electrode locations. For the seizure period there are clear differences between the estimation results between animals. For animals one to three the excitatory and fast inhibitory synaptic gain, as well as the input mean is always larger than it is prior to seizure. This is not the case for animal four where the excitatory and fast inhibitory synaptic gain as well as the input mean decreases at seizure initiation. The variation between seizures both in terms of trends and values converged to, is clear for the slow inhibitory synaptic gain. The results from animal one and four are more similar than the results for animals two and three. 

The results from the estimation procedure above suggest that the mechanisms involved in seizure may vary between animals, and possibly within animals over longer time periods. This contradicts the majority of studies that have shown single parameter sets that are capable of describing the transition from background to seizure~\citep{wendling2005interictal}. The reason for this is two fold: the first is that the majority of studies attempt to match the frequency of the observed EEG to the simulated model, whereas the unscented Kalman filter approach makes uses each individual recording and its covariance to determine the most likely parameter values that describe each observation. The second reason is that the unscented Kalman filter makes the assumption that the model is not capable of completely describing the observed EEG (model uncertainty), this uncertainty incorporates model inadequacy as well as the stochastic effect of the input on the model output. In this way the unscented Kalman filter is more robust at dealing with stochastic inputs.

The results comparing the standard deviation and mean of a single simulation compared to a Monte-Carlo simulation demonstrate that the unscented Kalman filter characterizes the expected error on each parameter with some accuracy~(Figure~\ref{fig: MonteResults}). In particular, when considering the slow inhibitory gain there is a high standard deviation in both the single and multiple estimation results prior to seizure. The standard deviation then decreases during seizure and slow increase post seizure. The results also show that the estimation technique is robust to the initialized parameter values. However, one interesting aspect to notice is the high standard deviation of the slow inhibitory gain prior to seizure. This is of particular interest as the estimation results from four seizures from this animal showed variations in the slow inhibitory synaptic gain prior to seizure. The variation between these estimated parameters appear to be due to the particular initializations of the model parameters.

\red{Reason for lower standard deviation for monte carlo analysis during seizure periods}
This may be due to a higher signal to noise ratio in seizure periods. Another possible reason is that this model was developed to mimic seizure dynamics. Therefore, in seizure periods there is less uncertainty about the model predictions then there would be in background EEG.
\red{end}

\red{Reason for high standard deviation post seizure, monte carlo}
 This is due to the nature of the model, and the estimation results for the excitory synaptic gain. Post seizure the excitory synaptic gain is lower than any other period in the estimation procedure. When looking at the structure of the model, it is clear that when the excitory synaptic gain is low the relative contribution of all other populations to the model output is low. With the correction step of the unscented Kalman filter the contribution of these populations would be seen to be similar to noise on the observations, therefore, there is higher uncertainty about the estimates. The reason for a high variance on the input is apparent, as it is the mean of a stochastic element. 
\red{end}

The Kalman filter is a Markov process, so there is an assumption that all the required information to determine the most likely states at the next time point can be compressed into the mean and covariance of the current states. Further, the greater the number of states that are used in the unscented Kalman filter framework the more likely the algorithm is to diverge from the values that best describe the observations. 
% 
% \section{Conclusion}

In this paper we have demonstrated that the tracking of neural dynamics using an unscented Kalman filter is feasible. We have also shown that there are clear differences between the mechanisms involved in seizure initiation and termination between and in some cases within animals. This information could be used to help with the development of new therapies. Some of the aspects that were not discussed in this paper include variability due to severity of seizures, and the long term evolution of seizures in an animal model where seizure frequency decreases over time. 
% 
% \section{Appendix}
\label{sec: AppendixA}

The reduction of the Wendling model is based on the principles of linear superposition. First notice that each synapse is a linear lowpass filter (equation~\ref{eqn: LaplaceNMM}). In figure~\ref{fig: Biological} there are eight such synapse. However, notice that numerous of these synapse have identical synaptic gains, and are connected by different connectivity constants. Given that for a linear system it is known that
\begin{align}
f(a\mathbf{x}) = af(\mathbf{x}),
\end{align} where $f(\cdot)$ is an arbitrary function, and $a$ is a constant. Therefore, the model can be simplified by moving the synapse closer to the respective populations. The result of this process is demonstrated in figure~\ref{fig: BiologicalMin}.

\begin{figure}  %%%%%%%%%%%%%%%%%%%%%%%%%%%%%%%%%%%%%%%
	\centering
		\includegraphics{Biological_Model_DescriptionMin.png}
	\caption{Graphical Description of the Wendling model (Minimised). Membrane potentials are shown and named $v_{b}$ where $b$ is $p$, $e$, $fi$ and $si$ for pyramidal, excitatory and slow and fast inhibitory populations, respectively. The synaptic gains of each population are specified by $\theta_{m}$ where $m$ is defined in the same manner as the membrane potentials and $\theta_{p}=\theta_{e}$. Each triangle and circle indicates a neural population. In particular, the triangle shape indicates the pyramidal population and the circle shapes represent interneurons. The triangles and circles model the action of the soma for the neural populations, and convert membrane potentials to firing rates. The synapses convert the firing rates to membrane potentials. Lastly, each line indicates a neural connection, which is specified by a connectivity constant.}
	\label{fig: BiologicalMin}
\end{figure}%%%%%%%%%%%%%%%%%%%%%%%%%%%%%%%%%%%%%%%%%%%%%

Now consider the full set of original model equations that would be required to specify figure~\ref{fig: Biological}
\begin{align}
\label{eqn: Fullmodel Descrip}
\mathrm{d}v_{po*}(t)&= z_{po*}(t)\mathrm{d}t\\
\mathrm{d}z_{po*}(t)&=\left(\frac{\theta_{p}(t)}{\tau_{p}}c_{1}g(v_{p}(t))-2\frac{z_{po*}(t)}{\tau_{p}}-\frac{v_{po*}(t)}{\tau_{p}^{2}}\right)\mathrm{d}t\\
\mathrm{d}v_{p1*}(t)&= z_{p1*}(t)\mathrm{d}t\\
\mathrm{d}z_{p1*}(t)&=\left(\frac{\theta_{p}(t)}{\tau_{p}}c_{3}g(v_{p}(t))-2\frac{z_{p1*}(t)}{\tau_{p}}-\frac{v_{p1*}(t)}{\tau_{p}^{2}}\right)\mathrm{d}t\\
\mathrm{d}v_{p2*}(t)&= z_{p2*}(t)\mathrm{d}t\\
\label{eqn: Pyr}
\mathrm{d}z_{p2*}(t)&=\left(\frac{\theta_{p}(t)}{\tau_{p}}c_{5}g(v_{p}(t))-2\frac{z_{p2*}(t)}{\tau_{p}}-\frac{v_{p2*}(t)}{\tau_{p}^{2}}\right)\mathrm{d}t\\
\label{eqn: SI}
\mathrm{d}v_{p3*}(t)&= z_{p3*}(t)\mathrm{d}t\\
\mathrm{d}z_{p3*}(t)&=\left(\frac{\theta_{si}(t)}{\tau_{si}}c_{4}g(v_{si}(t))-2\frac{z_{p3*}(t)}{\tau_{si}}-\frac{v_{p3*}(t)}{\tau_{si}^{2}}\right)\mathrm{d}t\\
\mathrm{d}v_{p4*}(t)&= z_{p4*}(t)\mathrm{d}t\\
\label{eqn: SI1}
\mathrm{d}z_{p4*}(t)&=\left(\frac{\theta_{si}(t)}{\tau_{si}}c_{6}g(v_{si}(t))-2\frac{z_{p4*}(t)}{\tau_{si}}-\frac{v_{p4*}(t)}{\tau_{si}^{2}}\right)\mathrm{d}t\\
\label{eqn: FI}
\mathrm{d}v_{p5*}(t)&= z_{p5*}(t)\mathrm{d}t\\
\label{eqn: FI2}
\mathrm{d}z_{p5*}(t)&=\left(\frac{\theta_{fi}(t)}{\tau_{fi}}c_{7}g(v_{fi}(t))-2\frac{z_{p4*}(t)}{\tau_{fi}}-\frac{v_{p5*}(t)}{\tau_{fi}^{2}}\right)\mathrm{d}t\\
\label{eqn: Exc}
\mathrm{d}v_{p6*}(t)&= z_{p6*}(t)\mathrm{d}t\\
\mathrm{d}z_{p6*}(t)&=\left(\frac{\theta_{e}(t)}{\tau_{e}}c_{2}g(v_{e}(t))-2\frac{z_{p6*}(t)}{\tau_{e}}-\frac{v_{p6*}(t)}{\tau_{e}^{2}}\right)\mathrm{d}t\\
\mathrm{d}v_{p7*}(t)&= z_{p7*}(t)\mathrm{d}t\\
\label{eqn: WienerFull}
\mathrm{d}z_{p7*}(t)&=\left(\frac{\theta_{p}(t)}{\tau_{p}}\mu -2\frac{z_{p7*}(t)}{\tau_{e}}-\frac{v_{p7*}(t)}{\tau_{p}^{2}}\right)\mathrm{d}t + \frac{\theta_{p}(t)}{\tau_{p}}\epsilon(t)\mathrm{d}W.\\
\end{align} In these equations $dW$ represents a Wiener process and is required as $\epsilon(t)\sim N(0,\sigma)$, where $\sigma$ and $\mu$ (equation~\ref{eqn: Wiener}) describe the mean and variance of the stochastic model input. Further, $v_{po}(t) $ and $v_{p1-7*}(t)$ represent the membrane potential produced by each synapse and $z_{po*}(t) $ and $z_{p1-7*}(t)$ their derivatives. The inputs to each neural population are specified by $v_{b}(t) $ and $z_{b}(t) $, and are the membrane potential of the specific population, where $b$ takes the values of $p$, $e$, $fi$ and $si$ representing pyramidal, excitatory, and slow and fast inhibitory populations, respectively. Therefore $v_{p}(t) $ is the output of the model. All $v_{b}(t) $ can be described in terms of $v_{po}(t)$ and $v_{p1-7}(t)$ as follows
\begin{align}
\label{eqn: pop1}
v_{p}(t) &= v_{p6*}(t)+v_{p7*}(t)-v_{p3*}(t)-v_{p5*}(t)\\
\label{eqn: pop2}
v_{e}(t) &= v_{po*}(t)\\
\label{eqn: pop3}
v_{si}(t) &= v_{p1*}(t)\\
\label{eqn: pop4}
v_{fi}(t) &= v_{p2*}(t)-v_{p4*}(t).
\end{align} Notice that equations~\ref{eqn: Fullmodel Descrip}-\ref{eqn: Pyr} and equations~\ref{eqn: SI}-\ref{eqn: SI1} are identical except for the connectivity constants. Further, in this model it is assumed that 
\begin{align}
\theta_p = \theta_e\\
\tau_p = \tau_e.
\end{align} Therefore, equations~\ref{eqn: Exc}-\ref{eqn: WienerFull} are identical, but have different inputs. Next consider the effect of the states $v_{p0-7*}(t)$ and $z_{p0-7*}(t)$ on the model inputs. Note that since the effect of $v_{p6*}(t)$ and $v_{p7*}(t)$ are additive, they can be added prior to being passed through the linear filter since
\begin{align}
f{x} + f{y} = f{x+y},
\end{align} where f is defined above and $x$ and $y$ are arbitrary variables. Therefore equations~\ref{eqn: Exc}-\ref{eqn: WienerFull} can be simplified to the following:
\begin{align}
\mathrm{d}v_{p1}(t)&= z_{p1}(t)\mathrm{d}t\\
\label{eqn: Wiener1}
\mathrm{d}z_{p1}(t)&=\left(\frac{\theta_{e}(t)}{\tau_{e}}(\mu +n_{e}g(v_{e}(t))-2\frac{z_{p1}(t)}{\tau_{e}}-\frac{v_{p1}(t)}{\tau_{e}^{2}}\right)\mathrm{d}t + \frac{\theta_{e}(t)}{\tau_{e}}\epsilon(t)\mathrm{d}W.
\end{align}. By making this reduction the effect of both the excitatory population and the input have been incorporated in one potential. Therefore, $v_{p7*}(t)$ is no longer required in equations~\ref{eqn: pop1}. 

Next consider the equations that only have connectivity constants that are different. Notice that these connectivity constants scale the input to the function, and this scaling can be performed after the input is transformed by the linear function. Therefore, equations~\ref{eqn: Fullmodel Descrip}-\ref{eqn: Pyr} can be represented by a single equation such that
\begin{align}
\mathrm{d}v_{po}(t)&= z_{po}(t)\mathrm{d}t\\
\mathrm{d}z_{po}(t)&=\left(\frac{\theta_{p}(t)}{\tau_{p}}g(v_{p}(t))-2\frac{z_{po}(t)}{\tau_{p}}-\frac{v_{po}(t)}{\tau_{p}^{2}}\right)\mathrm{d}t.
\end{align} Notice that in doing this equations~\ref{eqn: pop2}-\ref{eqn: pop4} are no longer valid and need to be altered to include the connectivities that have been removed from their relevant functions
\begin{align}
\label{eqn: pop2n}
v_{e}(t) &= c_{1}v_{po}(t)\\
\label{eqn: pop3n}
v_{si}(t) &= c_{3}v_{p0}(t)\\
\label{eqn: pop4n}
v_{fi}(t) &= c_{5}v_{p0}(t)-v_{p4*}(t).
\end{align}. 

A similar argument can be applied to equations~\ref{eqn: SI}-\ref{eqn: SI1} which can be simplified to
\begin{align}
\mathrm{d}v_{p2}(t)&= z_{p2}(t)\mathrm{d}t\\
\mathrm{d}z_{p2}(t)&=\left(\frac{\theta_{si}(t)}{\tau_{si}}g(v_{si}(t))-2\frac{z_{p2}(t)}{\tau_{si}}-\frac{v_{p2}(t)}{\tau_{si}^{2}}\right)\mathrm{d}t.
\end{align} Again by doing so equation~\ref{eqn: pop1} and~\ref{eqn: pop4n} are no longer valid and become
\begin{align}
v_{p}(t) &= v_{p1}(t)-c_{4}v_{p2}(t)-v_{p5*}(t)\\
v_{fi}(t) &= c_{5}v_{p0}(t)-c_{6}v_{p2}(t).
\end{align}. Lastly for simplicity equations~\ref{eqn: FI}-\ref{eqn: FI2} are replaced with
\begin{align}
\mathrm{d}v_{p3}(t)&= z_{p3}(t)\mathrm{d}t\\
\label{eqn: FI1}
\mathrm{d}z_{p3}(t)&=\left(\frac{\theta_{fi}(t)}{\tau_{fi}}c_{7}g(v_{fi}(t))-2\frac{z_{p3}(t)}{\tau_{fi}}-\frac{v_{p3}(t)}{\tau_{fi}^{2}}\right)\mathrm{d}t.
\end{align} Which results in the set of equations demonstrated in the methods section. Notice that the $n_{b}$ term used in the general form of the equations specifies connectivities that were not removed from the full model description equations due to simplification.



\bibliographystyle{apalike}
\bibliography{RealData}


\end{document}

