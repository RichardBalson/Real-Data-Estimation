\section{Introduction}

\red{Aims}

\red{Estimate model parameters from a neural mass model of the hippocampus.}
At present, the mechanisms underlying the generation of epileptic seizures are unknown, and approximately one third of patients with epilepsy are refractory to treatment. In this paper, a new method to estimate physiological properties of the brain is introduced. This method involves the application of a well established neural mass model with a new estimation technique. In particular, the application of an unscented Kalman filter (UKF)~\citep{voss2004nonlinear} for estimation of physiologically relevant parameters from a neural mass model of the hippocampus~\citep{wendling2002epileptic} (referred to as the ``Wendling model'') using EEG recordings is considered. It is hypothesised that, using the UKF, parameters from the Wendling model can be tracked with simulated noisy observations.

\red{To improve the understanding of epilepsy and improve the treatment of epilepsy patients.}
Estimating physiologically descriptive parameters from the Wendling model will make it possible to approximate changes occurring in the brain that lead to seizures. This will be achieved by estimating model parameters based on electrographic recordings of seizures from the hippocampus. Further, this method can be applied to determine the effect that treatment has on the brain, by observing changes in the estimated model parameters. Approximating the brain's physiology in this manner will make it possible to titrate therapies that are patient-specific and optimise their efficacy. For example, if, in a specific patient, it is found that a specific parameter space is estimated for most seizures, then a treatment strategy that forces the model parameters away from this space can be determined.

\red{How has this been achieved previously?}

\red{Introduction to neural mass model, freeman, jansen etc}
Neural mass models, originally formulated in the early 1970's~\citep{wilson1973mathematical,lopes1974model,freeman1963electrical} and developed subsequently~\citep{jansen1995electroencephalogram}, describe a cortical region of the brain as having populations of inhibitory and excitatory neurons. The net dynamics of the soma and synapses of neural populations are modeled by two functions. The first function describes how a synapse reacts to an firing rate in terms of a propagation delay and a synaptic gain, and takes the form of an integral kernel. The delay specifies the time taken for action potentials to propagate from one population to the next through dendritic trees, and the synaptic gain is a measure of the membrane potential magnitude resulting from a single action potential arriving at the considered population. The second function describes how the membrane potential of each neural population is converted into a firing rate. The firing rate specifies the average number of action potentials generated from the considered population. The second function is modeled as a sigmoid, which was originally formulated to describe the probability of a neuron firing given a specific membrane potential. In this case, population dynamics are considered; therefore, the sigmoid output is a firing rate, which is dependant on a population's membrane potential. Lastly, the number of synaptic connections linking neural populations together is specified by a connectivity constant. 

\red{What is the output of the model and why}
Neural mass models are phenomenological models of EEG, that describe the dynamics of excitatory and inhibitory populations. Excitatory neurons primarily consist of pyramidal neurons \iref which are known to have similar orientations. The parallel orientation of the pyramidal neurons allows for the electrical fields that they produce to sum together. For other neural populations, orientation is random and their net effect on the measured electric field is minimal. Therefore, the excitatory neural population is the generator of observed EEG. 

\red{Is a neural mass model a good model}
Are neural mass models, good models of the brain? To answer this question a definition for the word model is required. In this paper a model is considered to be an approximate descriptor of a system, which in this case is the brain. Since neural mass models are derived from aspects of neural physiology they can be considered to be models of the brain. Whether they are good models of the brain depends on how well they are able to replicate the observations they are intended to mimic. That being said a good neural mass model is merely one of numerous descriptors of the brain, and without further evaluation of the results obtained from the model with physiological studies a model provides little evidence of casual relationships. However, by using these models insight can be gained into what aspects of physiology should be evaluated, and can provide a method for aspects of physiology that can  not normally be observed, to be crudely estimated. For this study, a good model will be considered to be any model that can describe normal and seizure EEG with distinctly different physiological properties. Further, this model should account for all relevant physiological studies that have demonstrated causal physiological effects that would affect the considered model of the brain. 

\red{Inadequacy of jansen model and intro to the wendling model}
Neural mass models have been shown to be capable of reproducing key phenomena observed in EEG. A model proposed by ~\cite{jansen1995electroencephalogram} was shown to be capable of replicating normal EEG as well as alpha waves by altering a subset of its model parameters. Further studies have shown that by altering different parameters the model could replicate almost all activity observed in EEG. When considering the hippocampus, a study performed by \cite{white2000networks} showed that within the hippocampus the effect of inhibition on the pyramidal population had two distinctly different propagation delays, and that both were significant for the reproduction of EEG. They hypothesised that the cause of the two different propagation delays were due to the location of the synapses connecting the inhibitory and pyramidal populations. Longer propagation delays are due to synapse connections far from the soma (peri-dendritic), and shorter delays are due to connections near the soma (peri-somatic). The different propagation delays of inhibitory populations in the hippocampus was incorporated into the neural mass model by \cite{wendling2002epileptic}. To account for the two propagation delays \cite{wendling2002epileptic} described two different types of inhibitory populations: one fast (peri-somatic), and the other slow (peri-dendritic). In the same study, it was shown that the addition of the peri-somatic inhibitory population made it possible to replicate almost all types of observed EEG by altering three model parameters. 

\red{This model is capable of replicating key characteristics observed in EEG prior to and during seizure.}

The Wendling model is capable of replicating key features observed in EEG prior to, during, and after seizures. This is achieved by altering physiological parameters that describe the balance between excitation and inhibition in the modeled region of the brain. Due to its description of neuronal connections and systems in terms of neural populations, the model only has ten parameters, of which three describe the balance between excitation and inhibition \citep{wendling2002epileptic}. By altering the three parameters describing the balance between inhibition and excitation almost all phenomena in EEG can be mimicked. Therefore, to imitate the observed output of iEEG, it is necessary to be able to estimate these three model parameters.

\red{Is the Wendling model a good model}
The Wendling model is considered to be a good model of the hippocmapus since it is capable of mimicking the key features required for this study, and has a strong link to the physiology in the hippocampus. One further advantage of this model is that only three parameters need to be altered to imitate EEG, which will allow for more accurate estimation. The reason for this is that as the number of parameters estimated increases so does the complexity and inaccuracy of estimation. This is in particularly important when considered estimation of real data, where the model is merely an approximation of the observations. If the number of parameters is large for this estimation there is bound to be large errors, due to numerous local minima in the cost function.

\red{Previous work on estimating the neural mass model of the hippocampus has been done using a genetic algorithm.}

\red{Estimation of the neural mass model (Genetic Algorithm)}

Estimating the model parameters in the Wendling model is non-trivial. This is due to the nonlinear structure of the sigmoid function, as well as the stochastic input in the model. The neural mass model of the hippocampus has previously been estimated using the genetic algorithm~\citep{wendling2005interictal}, which is capable of estimating model parameters iteratively. The iterative procedure ensures that the genetic algorithm converges, although it may converge to the incorrect parameters. For the the genetic algorithm to converge, the data analysed needs to be stationary, i.e. model parameters need to be constant over the considered period of observations. 

\red{Kalman filter}
Another estimation method known as Kalman filtering allows the estimation of parameters in real time. The Kalman filter updates model parameters based on each observation made, and approximate the most likely model parameters that could have resulted in the current observation. However, the Kalman filter can only be applied to linear system estimation. This led to the development of new estimation techniques derived of the Kalman filter. One method is the UKF which approximates the system nonlinearity \iref. This approximation has been shown to be accurate in many studies \iref. 

\red{Why the Kalman filter}
The advantage of the UKF over the genetic algorithm is that there is no requirement that the observations are stationary, as it can track the changes in model parameters. This is important as subtle changes in model parameters may give an indication of when a seizure is about to occur, and could provide evidence of the effect of therapeutic treatments on the brain's physiology. These features of the UKF may allow for it to be implemeted in applications such as seizure prediction and responsive stimulation This is not possible using the genetic algorithm as it is an iterative estimation technique.

\red{What is being done, and why is it better or different?}

In this paper, the application of the UKF for the estimation of the three model parameters describing the balance between excitation and inhibition is considered. Initially, EEG is simulated using the Wendling model, which is then used as the observations for the estimation procedure. Model parameters are then estimated, under the assumption that they were originally unknown. The robustness of the estimation procedure is determined by evaluating the accuracy of estimation under conditions where observation noise is varied and states are initialised with of errors from their actual values.

\red{What is being done with the UKF}
The UKF, unlike the genetic algorithm, does not rely on iteration and is less time consuming. This computational efficiency comes at the cost of accuracy. This paper looks at the accuracy of the filter under numerous conditions to determine how robust it is. If the UKF is accurate at tracking model parameters then this method could be used to help characterise full EEG data sets, and allow for treatments to be evaluated and developed. This method may allow for patient specific treatments to be developed.

\red{Structure of the paper}
In the methods section, a description of the neural mass model of the hippocampus is presented, as well as the equations used to simulate the model. Further, the formulation of the UKF for the Wendling model is described. In the results section, the performances of the algorithm under numerous conditions are demonstrated. Lastly, in the discussion section, an evaluation of the performance of the filter is provided, discussing whether this method is a viable way forward to use model estimation to help approximate the effect that disorders and treatments have on the brain. 

\red{Model estimation and accuracy for real data}