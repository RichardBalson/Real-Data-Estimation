\section{Discussion}

\red{What does this study add}
Epilepsy is a poorly understood disorder, and there is uncertainty about the mechanisms involved in seizure generation and termination. However, with computational models of neural recording becoming more descriptive of physiology, it is becoming possible to gain insights into the underlying mechanism involved in recorded EEG. Computational models of EEG are non-linear, and standard estimation techniques cannot be used to approximate the physiological states from these models. In this paper, the application of the unscented Kalman filter to the Wendling model has been considered to help further the understanding of the mechanisms involved in seizure generation and termination.

\red{Why neural mass model, and is it relevant to physiology}

Is the Wendling model a good model of small neural population dynamics? To answer this question a definition for the word model is required. In this paper a model is considered to be an approximate descriptor of a system. Since neural mass models are derived from aspects of neural physiology they can be considered to be models of the small neural population dynamics. Whether they are good models depends on how well they are able to replicate the observations they are intended to mimic. That being said, a good neural mass model is merely one of numerous descriptors of the small neural population dynamics, and without further evaluation of the results obtained from the model with physiological studies a model provides little evidence of casual relationships. However, by using these models, insight can be gained into what aspects of physiology should be evaluated, and can provide a method for aspects of physiology that can not normally be observed, to be crudely estimated.  

The relevance of computational models of neural masses to real physiology is still an open question, and we do not propose that the results shown here are indicative of actual physiological mechanisms. However, the derivation of the computational model considered is based on physiological observations made in the hippocampus\citep{freeman1963electrical,wilson1973mathematical,white2000networks}, and the ability of the model to mimic recorded EEG~(Figure~\ref{fig: EEG}) from the hippocampus is promising. If we assume that the the model is a good description of the cortical region we record EEG from, then we can at least say that the estimation results provide some insight into the most likely causes of seizures. If we further assume that variability in the underlying structure of recorded EEG is due to different physiological mechanism at play, we can also conclude that variability in the estimation results between, and within, animals is due to different mechanisms of action, and that seizures are indeed specific to each individual animal.

Figure~\ref{fig: EEG} demonstrates that the Wendling model can capture key characteristics of recorded EEG from the hippocampus of the tetanus toxin model of epilepsy. However, the model can not capture all the dynamics of the recorded EEG. So how is this difference between the model and the actual system incorporated into the estimation technique.

\red{Why estimation (UKF)}

The unscented Kalman filter can account for model inaccuracy by increasing the uncertainty in the predictions made by the model. This can be altered in equation~\ref{eqn: PriorSCov} by increasing the value of $\mathbf{Q}$. Increasing the uncertainty in the predictions from the Wendling model allows the unscented Kalman filter to track key dynamical changes in the recorded EEG~(Figure~\ref{fig: EEGComp}). This is further demonstrated by the changes in the dynamics of the estimated model parameters at seizure initiation and termination~(Figure~\ref{fig: EstimationResults}. 

\red{Differences between model simulation and recordings}

Although the estimation of model parameters from the Wendling model can track key structural changes in the recorded EEG, the simulated data using these parameters does not match the recorded EEG. The first reason for this is that the input to the simulated data varies from simulation to simulation because it is stochastic. Secondly, we assume that the stochastic input to the model has a constant variance, which may not necessarily be the case. In fact by increasing the variance to the simulated model recorded seizure activity is  more accurately replicated; however, pre-seizure and post seizure EEG are less accurately mimicked. Lastly, the uncertainty applied to the model can not be replicated in the simulation process. For the estimation procedure the uncertainty added to the model allows it to have dynamics which are not inherent in its original formulation. Therefore, when we attempt to simulate the recorded data using the estimated results, the best result we can expect is for there to be structural similarities between the recorded and simulated EEG~(Figure~\ref{fig: EEGComp}). 

\red{Uniqueness and convergence}

The next issue to consider is whether these captured transitions to and from seizure are unique, and if not are the estimates for the parameters the best description of the recorded data. When considering whether the transitions to seizure are unique, we have results demonstrating that they are not, and that different recorded EEG can be caused by different estimated model parameters. However, when considering whether the estimated parameters are the best description of the recorded data, the reliability of the estimates of each model parameter comes into question. The results shown for the pre-seizure model parameters are consistent, which indicates that the model is converging to the most likely parameters for this period. Further, the estimation results during seizure show that for particular animals there are similar trends in the model parameters~(Figure~\ref{fig: EstimationResults}). Lastly for the reliability of the estimates the Monte-Carlo simulation demonstrates that during seizure their is the least amount of variability in the estimation results. This indicates that the estimation results for the seizure period are affected the least by the initialisation of of model parameters and are quite reliable. 

\red{Similarities in estimation results}

The estimation results demonstrate similarities and differences for the mechanisms involved in seizure initiation and termination between and within animals. For animal one the pre-seizure estimates for all parameters, except slow inhibition, converge to similar values. This provides good evidence that the estimation is indeed extracting the key characteristics of the recorded EEG. Further, for animal one the estimated model parameters during seizure converge to similar values and have very similar trends. The post seizure results for slow inhibition and excitation are similar between seizures; this is not the case for the mean input firing rate and fast inhibition. Similar results are observed in the other three animals. For animal two pre-seizure results for all parameters except the fast inhibition synaptic gain are similar, were there is variability in the values the algorithm converges to during seizures; however, the trends in all estimated parameters for each seizure are alike. Animal 3 has very similar results to animal two in terms of parameters that converge to similar values and the trends in the model parameters.

\red{Variability within animals}

The results for animal four are different to the results from the other three animals. There is variability in all the estimated model parameters for all periods shown. The only parameter that converges to similar values between seizures is the excitory synaptic gain pre-seizure. Of particular interest is the estimation results from seizure three, here the model parameters converge to values that are physiologically implausible. There, are two possible reasons for this result: firstly, divergence in the unscented Kalman filter; secondly, model inadequacy when considering this particular seizure. The variability in the results for this animal could possibly be due to the framework considered in this study not being adequate to account for the dynamics observed in the recorded EEG. This could further demonstrate the variability in mechanisms involved in seizure initiation and termination. By this I mean that for the first three animals the framework developed appears to be able to capture the key dynamical changes in the recorded EEG; however, for animal four this does not always appear to be the case. Therefore, it is plausible that there are changes occurring in the brain that can not be captured by this framewrok which may lead to seizure initiation and termination.

\red{Variability between animals}

The results from the estimation procedure above suggest that the mechanisms involved in seizure may vary between animals, and possibly within animals over longer time periods. This contradicts the majority of studies that have shown single parameter sets that are capable of describing the transition from background to seizure~\citep{wendling2005interictal}. The reason for this is two fold: the first is that the majority of studies attempt to match the frequency of the observed EEG over 'stationary' periods to the simulated model output, whereas the unscented Kalman filter approach makes uses each individual recording and its covariance to determine the most likely parameter values that describe each observation. The second reason is that the unscented Kalman filter makes the assumption that the model is not capable of completely describing the observed EEG (model uncertainty), this uncertainty incorporates model inadequacy as well as the stochastic effect of the input on the model output. In this way the unscented Kalman filter is more robust at dealing with stochastic inputs, and data that it can not accurately mimic.

\red{Certainty about results}

The results comparing the standard deviation and mean of a single simulation compared to a Monte-Carlo simulation demonstrate that the unscented Kalman filter characterizes the expected error on each parameter with some accuracy~(Figure~\ref{fig: MonteResults}). In particular, when considering the slow inhibitory gain there is a high standard deviation in both the single and multiple estimation results prior to seizure. The standard deviation then decreases during seizure and slow increase post seizure. The results also show that the estimation technique is robust to the initialized parameter values. However, one interesting aspect to notice is the high standard deviation of the slow inhibitory gain prior to seizure. This is of particular interest as the estimation results from four seizures from this animal showed variations in the slow inhibitory synaptic gain prior to seizure. The variation between these estimated parameters appear to be due to the particular initializations of the model parameters.

\red{Reason for lower standard deviation for monte carlo analysis during seizure periods}
This may be due to a higher signal to noise ratio in seizure periods. Another possible reason is that this model was developed to mimic seizure dynamics. Therefore, in seizure periods there is less uncertainty about the model predictions then there would be in background EEG.

\red{Reason for high standard deviation post seizure, monte carlo}
 This is due to the nature of the model, and the estimation results for the excitory synaptic gain. Post seizure the excitory synaptic gain is lower than any other period in the estimation procedure. When looking at the structure of the model, it is clear that when the excitory synaptic gain is low the relative contribution of all other populations to the model output is low. With the correction step of the unscented Kalman filter the contribution of these populations would be seen to be similar to noise on the observations, therefore, there is higher uncertainty about the estimates. The reason for a high variance on the input is apparent, as it is the mean of a stochastic element. 

\red{Why is this useful}

The framework proposed here, demonstrates variability in the mechanisms involved in seizure initiation and termination. This is of particular interest when considering the fact that patients with similar epileptic pathologies respond to the same treatments differently \red{ref}. If this estimation framework allows us to capture the differences in the mechanisms involved in seizure initiation and termination, this information could be used to determine what therapy may be effective at treating individual patients. A further possibility with this technique is tracking estimated physiology before and after treatment and observing what dynamical change in the model parameters provides the best prognosis for the patients. By doing so this technique could possibly be used to develop new therapies, or titrate stimulation paradigms.

\red{Shortcomings}

The framework proposed has shortcomings, some of which have already been discussed. One particular aspect which has not yet been discussed in the structure of the estimation technique. The Kalman filter is a Markov process, so there is an assumption that all the required information to determine the most likely states at the next time point can be compressed into the mean and covariance of the current states. Further, the greater the number of states that are used in the unscented Kalman filter framework the more likely the algorithm is to diverge from the values that best describe the observations. This is part of the reason for the selection of this particular model which requires only four parameters and eight states to mimic intracranial EEG.

